\subsection{Sea Movement and Battles}

Black lines represent straits, blue lines represent sea lanes.

There are three sea areas - Western, Central and Eastern Mediterranean. The Atlantic and the Black Sea are not sea areas in this game. The only movement allowed in the Atlantic is between Lugudenensis Gallia and Britannia via straits.

If you control a province that borders at least one of the Mediterranean sea areas then you can attack any land province that is adjacent to \textbf{any} of the three sea areas.

Roll a die. On a roll of 4 or more you succeed and continue with your attack. On a 3 or less the attempt has failed and your action turn ends. You gain a +1 die modifier for attacking along a sea lane and a further +1 modifier if the area/province being attacked is adjacent to a sea area that you control.

\textit{EXAMPLE: If you wish to attack Britannia then you would first need to make a die roll for sea movement. You would have to roll 3 or higher, gaining a +1 modifier for the sea lane. If you succeeded then you would attack Britannia, which would require another die roll if neutral.}

\subsubsection{SEA BATTLES}

If you want to take control of a sea area from another player then you will have to fight a sea battle. You first roll for sea movement and, if successful, instigate a sea battle.

Sea battles work in a similar manner to a land battle but with a number of restrictions. It is only possible to select either one Sword or one Spear unit to fight (ship-borne soldiers), one Foot Skirmisher or Archer (ship-borne skirmishers), and one Artillery unit.

You can use any number of galley cards to bring the number of cards up to five. Cards are played in any order, since there are no fast or slow units in sea battles. For the purposes of Artillery fire all targets are regarded as being galleys. Artillery takes precedence over Skirmishers and/or Archers.

It may occur that you cannot play as many cards as your opponent. Both players still play cards as normal to start, but eventually you will not be able to play cards. You lose a combat round for each card you cannot match, i.e. the other player moves a battle block to their side of the display.

If you win then you take control of the sea area, and may also take plunder. The Rout rule is also in effect.

Note that control of a sea area doesn't allow you to stop somebody else from attacking through it. It only means the other player cannot gain the +1 modifier.