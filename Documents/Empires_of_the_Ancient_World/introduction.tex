\section{GAME OVERVIEW}

Empires of the Ancient World is set in the Mediterranean world between 200 B.C. and 200 A.D. Each player’s goal is to conquer as much of the known world as possible, not just militarily but also through trade.

The heart of the game is the card system. Players decide the tactics of their armies by recruiting cards. Each type of card has a strength and a weakness. For example, Pike units will beat Cavalry but will suffer at the hands of sword wielding Infantry. Skirmishers can be used to attempt to screen units but must be wary of being hit by Cavalry. Players can also engage Merchants to increase their trading capacity, use Engineers to build Fortifications, and Diplomats to influence neutral provinces.

The game is for 3 to 5 players and should take between 2 and 3 hours to complete.

\subsection{Components}

\begin{itemize}[nosep]
  \item Rule book
  \item Map
  \item Empire Army cards (yellow and green borders)
  \item Available Army cards (red borders)
  \item Player Control markers(disks)
  \item Player Trade markers (cubes)
  \item Fortification markers (black disks)
  \item Battle blocks (black cylinders)
  \item Two six-sided dice
  \item 1st Player marker (wooden pawn)
  \item Round and Tum Markers (black cylinder)
\end{itemize}

\subsection{Common Words and Phrases}

If the rules say "roll a die", it means roll 1d6. If the rules say "roll the dice", it means roll 2d6.

If the rules say "you", it's generally referring to the active player. It's you, you're the player.

\subsection{Cards trump Rules}

In any case where the cards conflict with the rules, the cards are considered correct.

\textit{Note that there is a Foot Skirmisher without its text box. This was a production error. Text from other Foot Skirmisher cards applies.}