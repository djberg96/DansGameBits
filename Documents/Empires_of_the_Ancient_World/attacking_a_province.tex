\subsection{Attacking a Province}

You can attack any province or sea area that is adjacent to a province or sea area you already control. You simply state which province or sea area you wish to attack.

\subsubsection{ATTACKING A NEUTRAL PROVINCE}

A province that does not have a control marker in it is regarded as being neutral. Roll a die to see if the you place a control marker in the province. On a 3 or higher the attack succeeds and you place a control marker in the province.

If the attack fails then the action phase proceeds to the next player, i.e. the only penalty for failure is the loss of an action.

\subsubsection{ATTACKING A SEA AREA}

A sea area that does not have a control marker in it is regarded as being neutral. To take control of it you have to make a successful sea movement roll. This is covered in the rules concerning Sea Movement and Battles.

If you want to attack a controlled sea area then this will involve a sea battle. See the rules on Sea Movement and Combat below.

\subsubsection{ATTACKING ACROSS MOUNTAINS OR STRAITS}

If you are attacking a neutral or controlled province across either mountains or a strait then you must make a die roll to see if you complete the successful movement of your army. On a roll of 2 or more the attack proceeds normally. If you roll a 1, then your action turn ends immediately.

\subsection{Battle}

A battle occurs when you a province that is controlled by another player. Each player must now select five army cards from those they hold in order to fight.

\subsubsection{AMBUSH}

If the defender has a Military Leader card on display then he may choose to fight in terrain that will favour skirmishers. He declares whether he wishes to do so after both players have selected the cards they will play with, but not the order in which they will be played. The effect of an ambush is to give every skirmisher card (on both sides) a +1 die roll modifier. The cards affected by this are Foot Skirmishers, Archers, and Light Horse.

\subsubsection{FIGHTING THE BATTLE}

The goal of each player in battle is to win the most battle blocks, up to three. These start the battle in the 'Centre Ground’ box of  the Battle Display.

Players must arrange their cards in the order they will be played, with the first card to be played at the top, last card to be played at the bottom. The cards are kept face down to keep them secret from the opposing player.

There are restrictions on how the cards are ordered. All fast cards must be played before slow cards.The only slow cards are Sword, Pike, Spear and Warbands. All other cards are regarded as fast. If a player has a Military Leader, then he can play one card out of sequence, e.g. one slow card before one fast card or one fast card after a slow card.

Players now simultaneously reveal the top card from their pile. The unit with the highest value wins the combat. The winner moves one of the battle blocks to their side of the battle display (either the attacker's or defender's side). If there are no blocks left in the center ground, then the player moves one block from the opposing player's side back to the center. If both units have the same value then the combat is tied and no block is moved.

Players then reveal the next card, resolving each battle, until each player has revealed all five cards. The player who has the most battle blocks on their side is the winner.

\textit{Note that the terms “kill”, “block”, etc, do not mean that the card is removed or lost in any way, they are simply used to describe an outcome (win, lose or tie). Casualties only happen at the end of combat.}

To make combat more interesting units have certain modifiers depending on which unit they are facing, as described below.

\subsubsection{UNIT TYPES}

\textbf{SWORD} - If a Sword unit is drawn against a Warband or Pike unit then increase its value by 1. If opposed by an Elephant then the combat is tied, unless the Elephant rampages.

\textbf{PIKE} - If a Pike unit is drawn against a Cavalry or Heavy Cavalry unit then increase its value by 1. If opposed by an Elephant then the combat is tied, unless the Elephant rampages.

\textbf{SPEAR} - If opposed by an Elephant then the combat is tied, unless the Elephant rampages.

\textbf{WARBAND} - To find out the final strength of the unit the player rolls one die and adds it to its base strength of 3, giving a possible strength between 4 and 9.

\textbf{CAVALRY} - If drawn against a Sword or Warband then increase its strength by 1.

\textbf{HEAVY CAVALRY} - If drawn against a Sword or Warband then increase its strength by 1.

\textbf{ELEPHANT} - The owner must roll a die to see if the unit rampages. On a roll of 1 it rampages, automatically losing the combat. If drawn against another Elephant that also rampages then the combat is a tie. If drawn against Cavalry then the unit increases its strength by 1.

\textbf{FOOT SKIRMISHERS} - This unit does not have a strength. Instead, the owning player rolls a die to see what effect it has. On a roll of 4 or 5 the unit blocks the opposing unit, resulting in a tie. On a roll of 6 or more the unit kills the opposing unit, winning the combat. Against an Elephant the unit gains a +1 die modifier. Against Cavalry the unit suffers a -1 die modifier. Against Heavy Cavalry the unit suffers a -2 die modifier.

If both players play a Foot Skirmisher the result is a tie.

\textbf{ARCHERS} - Archer units work the same way as Foot Skirmishers except that they kill instead of block on a 5 or 6. If drawn against Elephants they gain a +1 die modifier. Against Cavalry they suffer a -1 die modifier. Against Heavy Cavalry they suffer a -2 die modifier.

If both players play an Archer the result is a tie. Archers and Foot Skirmishers also negate each other.

\textbf{LIGHT HORSE} - These are a form of mounted skirmishers. They block on a 4 or 5 and kill on a 6, the same as Foot Skirmishers. However, if they meet Foot Skirmishers or Archers then they automatically win. If drawn against Elephants they gain a +2 die modifier, against Cavalry they suffer a -1 die modifier, and against Heavy Cavalry they suffer a -2 die modifier.

\textbf{ARTILLERY} - The owner player rolls a die to see if the unit hits its target, which is considered a kill. Against infantry and elephants it must roll a 6 or more. Against a Galley, other Artillery, and any units involved in a Siege Combat, a 5 or more must be rolled. If the unit misses then it loses the combat

Artillery always misses Cavalry, Heavy Cavalry, Foot Skirmishers, Archers and Light Horse in Open Combat.

In Siege Combat the Artillery result overrides a Skirmisher result, i.e. if an Artillery unit hits then it automatically wins, regardless of what the Skirmisher result is, but loses otherwise, i.e. the Skirmisher does not need to roll.

If the owner has an Engineer card on display then he gains a +1 die modifier. If two Artillery units simultaneously hit each other then the result is a tie.

\textbf{SIEGE TOWERS} - Siege Towers can only be used in Siege Combat, in which case they have a strength of 9. If the player has an Engineer card on display then it has a strength of 10. Both the attacker and defender may play a Siege Tower in Siege Combat.

\subsubsection{Siege Combat}

If the defender has a Fortification in the province being attacked then the defender has the option to choose to fight from inside the fort, resulting in a siege combat. The defender may also choose to fight in the open instead, in which case the Fortification has no effect.

In Siege Combat the players only choose three cards each (instead of the normal five). The cards can be played in any order and there is no Ambush allowed. Neither side may use mounted units (Cavalry, Heavy Cavalry, Elephants and Light Horse).

\subsection{Winning the Battle}

At the end of the battle, after all the cards have been played, the player with the most battle blocks on their side wins the battle. If there is a tie then the defender wins. Both sides must now take casualties. The attacker and defender each randomly loses one card from those played for every battle block on the other player’s side.

Neither side can lose an Empire army card. If one is drawn then this still counts towards the casualty total but is retained by the player. Thus, it is possible for you to avoid losing an available Army card if you use a number of Army cards.

Each available Army card drawn as a casualty is placed in the discard pile. If the attacker won then remove all of the defender’s control markers in the province, and place one of the attacker's control markers there.

\subsubsection{ROUTS}

If, at any time during combat, one player has all three battle blocks on their side then the battle ends immediately in a rout.

If the attacker routs then defender, then he can take another action. This does not necessarily have to be an attack action. If the attacker achieves a second rout, he does not gain another free action.

If the defender routs the attacker then the defender takes 2 trade blocks from the attacker’s Warehouse, if available, and places them on his own Warehouse.

\subsubsection{PLUNDER}

If the attacker wins then any trade blocks in the province just captured that are not of his color are removed and placed on the attacker’s warehouse card. These will count as victory points.

\subsubsection{THE LAST PROVINCE}

A player cannot lose their last province, which means there is no point in attacking it.
