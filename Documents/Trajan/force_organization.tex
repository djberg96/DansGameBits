\section{FORCE ORGANIZATION}

Generally, all units belonging to the same player in a single hex are grouped together as a single force. They must move and conduct combat together. There are some exceptions, as noted below.

\subsection{Stacking}

Having more than one unit in a hex (or Battle Board square) is called "stacking”. There is no limit to the number of units which may be in a strategic map hex or in the game displays. Friendly and enemy units may be stacked together in the same hexagon.

Exceptions: there never may be more than one Civis unit per hexagon, nor may there be a friendly and enemy Civis unit in the same hex.

\subsection{Movement}

Only one movement attempt may be made out of a hex each turn. However, not all units in the hex need move, i.e. you can leave units behind in the initial hex.

When moving, a force may pick up/drop off units. See the Movement rules for a full explanation.

\subsection{Combat}

All units in a single hex must attack together regardless if they started as separate forces; i.e., if two separate forces moved into a hex, they are combined for combat.

A player must use all units in a force to conduct an attack. All defending units in a defending force must defend.

Normally, no force may attack more than once in a single combat segment. However, play of a Pursuit Military Stratagem allows multiple combats. See the Stratagem Marker summary in the Player Aid Card.

Generally, no defending force may be attacked more than once in a single combat segment unless the combat occurs in a city hex and a besieging force is attacked by the enemy relief force and garrison separately (see below) or as the result of a Pursuit.

\subsection{Cities}

A player who controls a city may have his units in the same hex either inside or outside of the city.

Units inside the city should be indicated by being placed underneath the Civis unit if present, or rotated 90 degrees if not. Units outside of the city should be placed on top of the Civis unit if present.

Civis units are always considered to be inside the city.

Civis units are not required to control a city. Non-Civis units are not required to be inside a city in order to control it. See section 9 for more details on control.

Units in a friendly controlled or unoccupied city hex may enter or exit the city itself during the friendly movement segment at no movement cost. Exception: friendly units may not move into or out of a city that is besieged by enemy units.

Units whose only move is into or out of the city must still check the March table. Movement within the hex costs 0 Movement Points.

Only units which are inside of the city get to use the city's supply capacity. Units in the same hex but outside of the city do not. (Advanced Rules)

\subsection{Siege Organization}

A siege occurs when one side controls a city and the other side has units in the hexagon. The city is then besieged.

If a city hex is besieged then the player controlling that city must divide any units he has in the hexagon into two stacks, the garrison and the relief force. Garrison units are units which were in the city prior to the enemy player moving into it (they are besieged). All friendly units that remained outside the city the instant an enemy force entered the city hex are considered the relief force.

During movement, garrison units may not move out of a besieged city hex, though garrison units (excluding Civis units) can attack besieging units via a sortie (15.3). Relief force units may move into and out of the hex normally, and join the relief force, but may not enter the city itself until all enemy units in the hex either leave or are eliminated.

Note that the “sortie” result on the siege table is not the same thing as exiting the city to fight a battle. Simply apply the results as indicated on the table.

\subsection{Siege Combat}

The besieging player may attack either the besieged city, or the relief force. He may not attack both in a single turn.

The besieged player may conduct two combats against a besieging enemy force. One combat may be conducted by the garrison, the other by the relief force. (Note this is an exception to the general rule that a force can only be attacked once per combat segment.)

Each attack is conducted separately in any order, and the results of one implemented before the second.

A garrison and relief force may not combine for an attack against a besieging force. This represents the historical problems of coordinating relief forces with besieged cities.

At the end of any attack by garrison units, garrison units may join the relief force if their side wins the battle (i.e., they may fight their way out of the city). If they lose the battle, they remain inside the city.

At the end of any attack by relief force units, relief force units may enter the city (and become part of the garrison) if their side wins the battle (i.e., they may fight their way into the city).
