\section{VICTORY CONDITIONS}

Trajan is won by controlling cities. Additionally, certain scenarios have their own specific set of victory conditions.

Victory is evaluated at the end of a scenario. Each player evaluates the number and types of cities he controls at the end of the game. Depending on the number and type of cities controlled, the players win various levels of victory.

There are three levels of victory. Gaining one condition is a Triumphant victory (lowest level of victory), two is a Conquering victory, and three is an Imperator victory (highest level of victory). The player who has the higher level of victory wins the game. If both players achieve the same level of victory, or neither player attains any level of victory, then the game ends in a draw.

\textbf{Roman Victory Conditions}

\begin{itemize}
  \item Control one or more Parthian regional capitals - Ctesiphon, Europas-Rhagae, Hecatompylus, and Ecabatana.
  \item Control both Armenian cities.
  \item Control at least forty cities (including those listed above).
\end{itemize}

\textbf{Parthian Victory Conditions}

\begin{itemize}
  \item Control one or more Roman provincial capitals - Antioch, Caesarea, Bostra, Mazaca, and Tarsus.
  \item Control both Armenian cities.
  \item Control at least twenty cities (including those listed above).
\end{itemize}

\subsection{Optimus Victory}

If the following conditions come into effect, the game immediately ends in an Optimus (Most Excellent) Victory for the player, as follows:

Romans, if they gain control of Ctesiphon, Europas-Rhagae, Hecatompylus, and Ecabatana for one complete turn (i.e., a friendly and enemy turn, complete).

Parthians, if they gain control of Antioch, Caesarea, Bostra, Mazaca, and Tarsus for one complete turn (i.e., a friendly and enemy turn, complete).

If one player concedes, the game ends in a victory for the other player.

\subsection{Treaty}

If both players agree, by mutual consent they may end the game in a Treaty, with neither side winning.
