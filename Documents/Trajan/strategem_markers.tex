\section{STRATAGEM MARKERS}

Each player has a set of 12 markers termed "Stratagem Markers." These represent various political and military capabilities. The explanation of effects of each type is on the Player Aid Card.

The number of Stratagem Markers each player starts with varies depending on the scenario.

Stratagem Markers may be retained until played. There is no limit to the number of Stratagem Markers that may be retained by a player. (Players may never have more Stratagem Markers than the counter limit.)

Upon playing, Stratagem Markers are returned to their side's Stratagem Marker pool (unless otherwise specified on the Player Aid Card).

\subsection{Play of Stratagem Markers}

Only leaders may play Stratagem Markers. A player may play a Stratagem Marker at any time in his own turn (Exception: Supreme Leader death).

For example, a player can move into a hex containing an enemy controlled Civis unit, convert it to his own side with a Political Stratagem Marker and continue moving.

Each player's leaders play Stratagem Markers in his own turn. Each leader may play a number of Stratagem Markers in his player turn equal to his leader rating. This should be noted separately on a piece of scratch paper.

Stratagem Markers must normally be played in the same hex the leader occupies. See below for exceptions.

In the Advanced Game, players may not play Stratagem Markers when on the Battle board.

\subsection{Play of Stratagem Markers on the Enemy’s Turn}

Leaders may play Stratagem Markers in the enemy's turn to counter play of enemy Stratagem Markers played against the hex containing the leader unit.

Only Stratagem Markers used to negate enemy Stratagem Markers may be played on the enemy's turn.

Each leader may play a number of Stratagem Markers in the enemy turn equal to his leader rating. For example, the Roman leader Trajan, with a Leader factor of 3, could play three Stratagem Markers in the Roman turn of a campaign month, and then three more in the Parthian turn of that month.

\subsection{Guard Units}

Certain unit types have an asterisk. Each Guard unit negates one Agent stratagem marker used for Intelligence or Assassination in the unit’s hex. Each Guard unit can only negate one Agent marker per game turn.

Historical note: these are units whose main purpose was internal security.

\subsection{Types of Stratagem Markers}

There are four types of Stratagem Markers - Military, Agent, Political, and Special.

Each side has a different mix of Stratagem Markers, representing underlying political and cultural differences between the Romans and Parthians.

Each type of Stratagem marker will have different types of actions a player can implement using it. For example, an Agent Stratagem marker can be used for either an Intelligence or Assassination action.

Certain Stratagem actions may be used only in the Advanced Game.

The Player Aid Card details the specific results of play for each type of Stratagem Marker.

\subsubsection{Stratagem Marker Example}

The Romans have one political and two military stratagem markers. Trajan and a Roman force are marching cross country. They enter a hex with a Parthian controlled city. Trajan decides to play the Political Marker as a Revolt.

He rolls a "2," indicating that the Civis unit is flipped to the Roman side. Since the city is now Roman controlled, the Romans may continue marching. At the completion of their march, the Roman player decides he wants the force to Force March, so he expends a Military Stratagem for this purpose. Trajan's force then enters a hex containing a Parthian force. Trajan decides to attack. He expends the remaining Military Stratagem marker to give him Tactical Superiority in this battle.

\subsection{Receiving Stratagem Markers}

Players receive Stratagem Markers at random, unless otherwise specified, for the following game events.

\subsubsection{Supreme Leader}

At the beginning his turn a player receives a number equal to the leader rating of his supreme leader. The player may examine and select the desired Stratagem Markers, i.e. these are not randomly selected.

If a player has no Supreme Leader, then he receives none under this provision, though may still receive them for other reasons.

\subsubsection{Winning battles or sieges}

A player who wins a battle, siege or revolt immediately receives a number of random Stratagem Markers depending upon the size of the victory - two for winning a Major Victory, one for a Minor Victory, and none for a Skirmish.

Note that cities taken as a result of the play of political markers, or units eliminated by Sedition and other non-battle/siege/revolt events do not count as victories under this rule.

See the Combat Rules for the definition of victory in battle.

\subsubsection{Roman Triumph}

The Roman Supreme Leader may play the Triumph Stratagem Marker (if possessed) in any Parthian capital. Depending upon the capital it is played in, the Romans will receive a certain number of Stratagem Markers per Roman Stratagem marker segment for the remainder of the game as long as the Triumph marker remains on that capital:

\begin{itemize}
  \item Ctesiphon: 1
  \item Ecabatana: 2
  \item Europas-Rhagae: 3
  \item Hecatompylus: 4
\end{itemize}

\subsubsection{Parthians capture a Roman provincial capital}

Parthians immediately receives one Stratagem Marker. This can only happen once per capital per game.

\subsubsection{Pillaging (Advanced Game)}

If a player pillages a city hex (regardless of who originally controlled it) both he and his opponent immediately receive a Stratagem Marker.

\subsubsection{Other Events}

Certain political events and the optional Historical Cities Events will give players Stratagem Markers.

\subsection{Losing Stratagem Markers}

Players lose Stratagem markers for the following game events:

\subsubsection{Elimination of Supreme Leader}

If a player's supreme leader is eliminated, then all Stratagem Markers he currently possesses are immediately lost, including the “Triumph”, “Trade Concession”, “King of Kings” and “Imperator” markers. New markers may be drawn for other game events as the game progresses.

\subsubsection{Losing battles or sieges}

A player who loses a battle, siege or revolt immediately loses a number of Stratagem Markers depending upon the size of the defeat.

The player chooses at random the appropriate number, of Stratagem Markers to be lost. This will be two for a Major Defeat, one for a Minor Defeat, and none for a Skirmish.

Cities taken as a result of the play of political markers, or units eliminated by Sedition and other non-battle/siege/revolt events do not count as victories under this rule.

\subsubsection{Events}

Certain Political Events and Historical Cities Events (see optional Historical Cities rule) will cause the loss of Stratagem markers.

\subsection{Stratagem Marker Limit}

The number of Stratagem Markers is a limit. If a player has all of his Stratagem Markers in his hand, and is required to draw more, then he may not do so.

If a player is mandated to lose Stratagem Markers, but has none remaining, then there is no additional penalty.
