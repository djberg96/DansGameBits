\section{SIEGES}

Siege Combat may be conducted in the Combat Segment against enemy units in cities. Siege combat only occurs when a player has units in the same hex as an enemy controlled city.

There are three types of siege combat a player may choose: Blockade, Assault, and Formal Siege.

\subsection{Siege Combat}

The active player must declare one of the three types of sieges for each friendly force which is in the same hex as an enemy controlled city in his Combat segment.

\subsubsection{Blockade}

When conducting a blockade, no attack is made. In the Advanced Supply rules, note that this means that the defender must roll on the city besieged line on the supply table if still besieged in his following supply segment.

\subsubsection{Assault}

This is performed exactly as regular combat, including tactical advantage, with the following exceptions:

There are only two round of combat, Fire and Melee. There is no Pursuit round.
Neither player may take a Maneuver Advantage.
Ignore all numerical combat losses to the defending units in the city. The only way to eliminate enemy units in an Assault is through disruption.
If the defender loses the battle, he does not retreat. Defending units remain in the city and retain control of it.

\subsubsection{Formal Siege}

In order to select Formal Siege the attacker must have had at least one Impeditus unit in the hex since the beginning of the turn. Thus, you cannot conduct a formal siege on the same turn that the Impeditus unit entered the hex, though you could Assault it.

A Formal Siege is similar to an Assault, except that the besieging player rolls once on the Formal Siege Table and immediately applies the results. Only roll once regardless of the number of Impeditus units present.

Siege Table Results are listed on the Player Aid Card.

\subsection{Defending Units}

Units defending in a city need not attack. They can sit the siege out. If they choose to attack, then the combat is resolved as per Battle rules, and the attacking units get no advantage for attacking from the city.

A Civis unit itself may not attack out of the city.

If units under siege win the battle they may leave the city. If they lose the battle, they must remain in the city.