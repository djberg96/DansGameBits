\section{COMBAT}

Combat takes place only between opposing units in the same hex on the strategic map. Unlike other war games, units in adjacent hexes may not attack each other. In the Basic Rules, Combat is resolved by use of the Basic Game Combat System (a cross referenced chart). In the Advanced game, Combat is resolved in a Grand Tactical manner by transferring the units to a Battle Board and refighting the battle itself.

Battle occurs when the player attacks enemy units which are not in a city. Battles occur in the Combat segment.

Siege occurs when the player attacks an enemy controlled city. Sieges occur in the Combat segment.

Revolt occurs when a Civis unit is flipped to its reverse side (via a Political Stratagem) and there are enemy units in the city. This happens immediately upon the Civis unit being flipped over.

Battle and Siege combat are not mandatory. It occurs only if the player whose turn it is declares he wants to attack. Otherwise, no combat occurs.

Revolt combat is mandatory.

\subsection{Civis units and combat}

Civis units will participate in combat in only two circumstances: defensively, when the city they occupy is attacked via siege, and offensively, when a Civis unit revolts.

\subsection{Basic Game Combat Displays}

There are two Basic Game Combat Displays: Battle and Siege. These are convenient places to place units when resolving combat. They have no effects on combat per se. Just place the units involved there for the battle and return them to the map when finished.

Units with a combat strength of 0 may not attack. They are eliminated if all other friendly units in a combat are eliminated.