\section{COMPONENTS}

\subsection{The Board}

The game map is divided into several sections: the Strategic Map, on which players maneuver their units, and various game displays which are used to resolve battles, organize units and perform other game functions.

\subsubsection{The Strategic Map}

The Strategic Map is the map of ancient southwestern Asia overlaid with a hexagonal grid. It based on the map that Claudius Ptolemy did of the region in the 2nd Century AD. It represents this region of the world, more or less, as the Romans saw it then.

The map is divided by hexagons, which are used to regulate placement and movement of units. There are several different terrain types as indicated on the Terrain Key on the map.

The following cities are Roman Provincial Capitals: Mazaca-Caeserea, Tarsus, Antioch, Caesarea, and Bostra.

The following cities are Parthian Regional Capitals: Ctesiphon, Ecabatana, Europas-Rhagae, and Hecatompylus.

\subsubsection{The Turn Record Chart}

This is used to determine the current year and month. Years are indicated in both Roman (AU=annum urbae, "year of the city [of Rome]") and Christian era dates.

\subsubsection{Terrain Key}

This shows the various types of Strategic Map terrain types.

\subsubsection{Basic Game Battle Display}

This is used as a convenient holding area for units which are resolving a combat in the Basic Game. When a battle occurs, simply transfer the units form the strategic Map to the Battle Display and resolve combat; when combat is finished, return the units to the map hex they originated in.

You are, of course, free to use the Advanced Game Battle Board as a holding area if you find it more convenient.

\subsubsection{Siege Display}

This is used as a convenient holding area for units which are involved in resolving a siege.

\subsubsection{Client Forces Holding Area}

This is used to hold Client Units which are not currently in play.

\subsubsection{Civis Holding Area}

This is used as a holding area for eliminated Civis units.

\subsubsection{Advanced Game Battle Board}

The Battle Board is used in the Advanced Game. It is used whenever enemy units engage in combat. It is divided into grid squares.

\subsection{Roman Charts}

Exercitus Holding Area - A holding area for units with corresponding Exercitus markers.

Recruit Holding Area - For units which are available for recruiting.

Off-Map Reinforcements Holding Area - For units on other parts of the Roman Empire.

Stratagem Marker Holding Area - Used to hold Roman Stratagem Markers.

\subsection{Parthian Charts}

Exercitus Holding Area - Used as a holding area for units in a corresponding Exercitus markers.

Core Recruit Holding Area - For Core units which are available for recruiting.

Satrapy Recruit Holding Area - For Satrapy units which are available for recruiting.

Stratagem Marker Holding Area - Used to hold Parthian Stratagem Markers.

\subsection{The Units}

The units in the game represent the historical forces and leaders which participated in the Parthian War. There are three sets of units: Roman, Parthian, and Client. The Roman player controls all Roman units. The Parthian player controls all Parthian Core and Parthian Satrapy units. Under certain circumstances, a player may also be able to control Client units.

Roman units are all red. The Parthians have three different types of units. The Parthian core units are blue; the Parthian satrapy units are green. Armenian, Arab and Alani client units are tan.

\subsubsection{Unit Information}

Each combat unit has the following information printed on it:

Name or identification (top)
A picture denoting the unit type (center)
Battle movement (right; advanced game only)
Combat strength (lower left)
Discipline (lower center)
Strategic Movement (lower right)

The name or identification is either the name of a leader, the historical unit designation, or a historical designation.

Leaders contain the same information, except that they also have a leader rating on the center left of the counter. See rule 2.6 below.

The unit type represents the class of unit. Within each class there may be specific subtypes.

\subsection{Unit Types}

\subsubsection{Heavy Infantry}

Legionaries - Roman sword and shield types.

\subsubsection{Light Infantry}

Auxiliary Infantry, a combination of missile armed troops and troops trained to fight in open order as well as Levies, Militia and rabble.

\subsubsection{Heavy Cavalry}

Parthian Cataphracts and Roman Heavy Cavalry. Used primarily for shock.

\subsubsection{Light Cavalry}

Equites, Javelin or Lance cavalry, used for maneuver and skirmishing and Equites Sagittarii, horse archers.

\subsubsection{Elephants}

\subsubsection{Civis Milites}

City Militia. These are referred to as “Civis” units.

\subsubsection{Impeditus}

Supply trains, siege engines, engineers, camp followers, etc.

Whenever the rules or charts refer to a general type it includes all types within that category. So, for example, if a rule refers to "Light Cavalry" it would include both equites and horse archers.

Legionnaires and Heavy Cavalry are the only heavy unit types. All others are considered light armed. This is indicated on the counters by giving those unit types a black unit symbol.

\subsection{Unit Ratings}

\subsubsection{Leader Rating}

This is the effectiveness of the leader on the strategic level, and will be 1, 2, or 3. This is the number of Stratagem makers the leader can play in a single turn. Additionally, certain non-leader units have an asterisk in this space: these are Guard units. This is explained in the Stratagem Markers rules.

\subsubsection{Strength}

The size of the unit: each strength factor represents around 1000 fighting men.

\subsubsection{Discipline}

This is the unit's training and morale. There are five types of discipline – I (Imperator), V (Veteran), R (Recruit), M (Mob) and B (Barbarian).

\subsubsection{Strategic Movement}

This is the basic number of hexagons a unit may move on the Strategic Map per turn.

\subsubsection{Battle Movement} (Advanced Game only)

This is the number of squares a unit may move on the Battle Board per turn.

\subsubsection{Unit Size}

Roman: All 5 strength heavy infantry are legions. Others are vexillations (battle groups).

Parthian: 1 strength units are "Dracos". Others are hordes.

All units are printed on their reverse side in their national color and symbol (Eagle standard for Romans, Draco standard for Parthians, sword for Clients). On the Strategic Map this is used for limited intelligence. In the Battle/Siege Displays and Battle Board it represents the unit in its disrupted state. The only exception is the Civis units; their front side (red colored) represents Roman control of a city, the reverse (green colored), Parthian.

\subsection{Game Markers}

\subsubsection{Stratagem Markers}

There are four types of Stratagem Markers

Military: used to influence battles.
Political: Used to cause enemy forces to switch sides.
Agent: used to conduct intelligence and assassination operations
Special: represents special cultural capabilities of each side.

\subsubsection{Exercitus (Army) Markers}

These markers are used on the strategic map to represent large stacks of units. Each Exercitus marker has a corresponding Holding Area on a player's charts.

\subsubsection{Battle/Siege Marker}

This is used to record the hex in which a battle/Siege is taking place (inasmuch as units conducting battles are temporarily transferred from the strategic map for combat resolution).

\subsubsection{Year/Month Markers}

The Year and Month marker designate the current turn on the Turn Record Chart.

\subsection{Game Tables} (Found on the Player Aid Card)

March Table: used to resolve movement.
Combat Results Tables: used to resolve combat.
Discipline Table: Used to resolve discipline checks.
Supply Table: used to determine effects of supply on units.
Political Events Table: Used to randomly generate a Political Event for a turn.
Stratagem Marker Summary: Details the effects of playing different Stratagem Markers.

\subsection{Game Scale}

Each Strategic map hexagon is roughly 400 stade or 50 Roman miles (74 kilometers) across; each strength point is 1000 men; each turn represents one month in summer or three in winter.
