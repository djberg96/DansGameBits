\section{RECRUITING}

Players receive additional units through Recruiting actions, which occur through the play of Political Stratagem Markers. Recruits include reinforcements (units which have not yet entered play) and replacements (units which were previously eliminated and are later returned to play).

\addtocounter{subsection}{1}
\subsubsection{Reinforcements}

Units which are designated as being part of a player's reinforcements start the scenario in their respective Recruit pool boxes. When brought into play, they are transferred to the map according to the specifications below.

\subsubsection{Replacements}

Units which are eliminated (for combat or other reasons) are placed in the player's respective Recruit pool boxes. They may be returned to play according to the specifications below.

Reinforcements and Replacements are received any time during the player's Movement segment via play of Recruit Political Stratagem Markers.

\subsection{Roman Forces}

The Romans have two groups of units they can recruit: Reinforcements and Off-Map Forces. Units available as Reinforcements and Off-Map Forces are designated by the scenario.

The Roman player raises reinforcement units through play of Political Stratagem Markers. A Roman selects a leader to play a Political Stratagem Marker, declaring a Recruit action. The Roman player receives one unit per Stratagem Marker so played. These are taken from the Roman Recruit holding area.

The leader playing the Stratagem Marker must be in a Roman controlled city during the Roman movement segment. Depending upon the discipline class of unit being recruited, there are further restrictions.

\begin{itemize}
  \item V class units may be recruited only in Roman provincial capitals.
  \item R class units may be recruited only in Roman cities (i.e., Red-tinted cities).
  \item M and B class units may be recruited in any Roman controlled city.
\end{itemize}

The Recruited unit is then placed in the same hex as the leader.

17.3. Roman Off-Map Forces

The Roman Supreme Leader may call in off-map reinforcements. The Roman Supreme Leader plays one Political Stratagem Marker. The Roman Supreme Leader then receives as many Off-Map reinforcements as are desired by the controlling player.

The quantity and type of Off-Map reinforcements are specified by the scenario.

Keep in mind that one of the events on the Political Events table calls for an off-map revolt. This is dependent upon the number of strength factors in the off-map Forces box currently. Thus, the more factors left in the Off-Map Box, the less the chance of an Off-Map Revolt.

Note that the Roman player may move units from the Strategic Map into the Off-Map Reinforcement Holding Area simply by conducting a successful March. It does not require a Political marker to move units into the Off-Map Reinforcement Holding Area. However, once units are moved into the Off-Map Reinforcement Holding Area, they may not be moved out unless the Roman Supreme Leader plays a Political marker. Also note that any type of Roman unit may be moved into the Off-Map Holding Area, not just units which started as Off-Map Reinforcements.

If the Romans have no Supreme Leader, they may not bring in off-map reinforcements. They may still march units into the Off-Map Forces holding area.

Roman reinforcements may be brought in on any turn, providing the above conditions are met.

\subsection{Roman Replacements}

Whenever a Roman unit is eliminated it is placed in the Roman Recruit holding area. Roman replacements are received exactly as reinforcements through play of Recruit Political Stratagem Markers.

Roman units which started as Off Map Forces and are eliminated are placed in the Roman Recruit holding area and may be taken as normal replacements (i.e., they are not returned to the Off-Map Forces box).

\subsection{Parthian Forces}

The Parthians control two groups of forces: Core and Satrapy. The Core units are blue, while the Satrapy units are green.

Parthian Core forces represent the Parthian central government and regular army. Satrapy units represent the various feudal forces which were raised by Parthian nobles.

\subsubsection{Parthian Core Reinforcements}

The Parthian player raises reinforcement units through play of Political Stratagem Markers. A Parthian leader plays a Political Stratagem Marker, declaring a Recruit action. That leader may then recruit any available Core unit.

Recruiting is done during the Parthian movement segment.

The Recruiting leader must be a Core (blue) leader. In addition, the leader must be in one of the Parthian Regional Capitals, which must be Parthian controlled.

The Parthian leader receives one unit per Stratagem Marker so played. The Parthian player may select which unit he will recruit from units available. The Recruited unit is then placed in the same hex as the leader.

\subsubsection{Parthian Core Replacements}

Core units which have been eliminated are placed in the Parthian Recruit pool. They are then eligible to be chosen as reinforcements.

The “Court” unit is considered a Parthian core unit and can be obtained as a reinforcement/replacement just like any other core unit.

\subsubsection{Satrapy Reinforcements}
The Parthian player raises Satrapy units through play of Stratagem Markers.

A Parthian leader (either Satrapy or Core) must be in a Parthian controlled city hexagon. This must be a city which was Parthian originally (i.e. green tinted).

The leader expends a Political Stratagem Marker and then draws at random one Parthian Satrapy unit from the reinforcement pool.

No more than one Parthian leader per city may raise units in a single turn in that city, but if a leader has a leader value of two, he may expend two Political markers and then recruit two units.

\subsubsection{Satrapy Replacements}

Satrapy units which have been eliminated are placed in the Parthian Satrapy Recruit holding area. They are then eligible to be chosen as reinforcements.

\subsection{Civis Units}

Civis units are an exception to the above recruiting/replacement procedure.

When a Civis unit is eliminated, it is placed in the common replacement pool. It may be replaced by either side. Whichever side replaces it may place it with its own side up.

In order to replace a Civis unit, a player selects a leader, expends a political Stratagem Marker, and then places a Civis unit with their side up.

Players may place Civis units only in cities they control and which have no other Civis unit. There never may be more than one Civis unit in a hex.

\subsubsection{Colonies}

The Roman player may, in his turn, exchange a Veteran legionary (heavy infantry) unit in the city for a Civis unit. This may be done only if the Roman player has a leader in the hex. The legionary unit exchanged in this manner is returned to the replacement pool and may be taken as a replacement as per the normal replacement rules.

\subsection{Leader units}

Certain leaders begin the scenario in their side's Recruiting pool. Off-Map leaders are recruited as per the normal recruitment rules. However, once eliminated, they are permanently removed from the game

Note that some scenarios start with certain leaders already eliminated.

The Parthian Court may be replaced if eliminated, since it is not considered a leader unit.