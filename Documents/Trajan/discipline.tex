\section{DISCIPLINE}

All units have a discipline rating. This has an impact on combat and other game functions, as explained in the appropriate rules.

\subsection{Types of Discipline}

There are five types of discipline:

\begin{itemize}
  \item I: Imperator - highest possible.
  \item V: Veteran - trained and battle experienced troops.
  \item R: Recruit - trained but inexperienced troops.
  \item M: Mob - untrained rabble in arms.
  \item B: Barbarian - undisciplined but fanatical.
\end{itemize}

\subsection{General Procedure For Discipline\\*Checks}

Roll one die per force and cross index the die roll with the respective discipline types of units belonging to the force. Apply the result to all units of each discipline type.

Example: A Roman force contains two Veterans and one Recruit Discipline Class units. The Roman player is required to make a Discipline Check in combat. He rolls a "3" on the discipline Table, and cross indexes the result: For Veterans, it is a P so the Veteran units pass; for Recruit Class units, the result is G, so the Recruit unit goes berserk.