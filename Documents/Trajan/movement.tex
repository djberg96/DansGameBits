\section{MOVEMENT}

In order to move, a force must roll on the March table. The procedure is as follows:

a. Designate the force to be moved.
b. Designate the type of movement it will use (road, cross country, river, sea).
c. Roll on the appropriate line of the March table.
d. Apply the result.

\subsection{March types}

Road: units moving by road march must move all of their movement on continuous road/trade route hexagons. They may not, for any reason, move off of them.

Cross country: units moving by cross country may move through any type of hexagons, except those explicitly prohibited to them.

River: units moving by river may move only along continuous river hexagons (representing use of boats).

Sea: The units may move from any coastal hex to any other coastal hex in the same sea. The force may not end its turn on an all sea hexagon. Sea movement may only be attempted if the player has at least one friendly coastal city on that sea. The units do not have to start or finish their movement in that city, it merely has to be there (representing basing for fleets).

\subsection{Forces}

A “force” is defined for movement purposes as all the units in a single hexagon which desire to move. No more than one force per hexagon may move per turn. The force does not have to include all units in the hexagon, but no more than one stack may move per hexagon per turn. All units in a force must move together. A force may drop off units along its route of march, but those units may not move any further that turn.

A force may pick up additional units as it marches. These units may not have moved separately in that turn, and can move only as far as the original force does.

The march factor of a force is equivalent to that of the lowest march factor in that force; however, if a force drops off its slower moving units, remaining units may continue to march up to their movement factor.

\subsection{Movement Limitations}

A force must stop in the first mountain, swamp or river hexagon it enters, except under the following conditions:

\begin{itemize}
  \item The force is moving along a road/trade route.
  \item The force includes an Impeditus unit (representing engineers clearing roads, building bridges, etc.).
  \item The unit is moving by river movement (even if moving through mountains or swamps).
\end{itemize}

A force must always stop when entering a hex containing an enemy unit.

A force which starts its movement in an enemy-occupied hex may leave.

If the moving force eliminates or politically converts all enemy units in the hexagon it moves into by playing Political Stratagem chits (see Stratagem chits rules) then it may continue moving.

Units may never enter all sea hexes, or cross all-sea hex sides, unless moving by sea movement. They may enter coastal hexes normally.

A March result of “A” (attrition) causes the loss of one unit in the moving force. This may include units that were picked up/dropped off at any point.

A March result of “W” (winter attrition) causes the loss of one half of the strength points in the moving force. Fractional losses are rounded up. This is figured from the total of all non-leader units which moved with the force, including units which were picked up or dropped off at any point.

\subsection{Forced March}

Play of a Forced March Military Stratagem allows a force to attempt movement twice in a turn (i.e., it may roll twice on the movement table). See the Stratagem Marker summary in the Player Aid Card.

A force may declare Force March at the end of its initial movement if a leader accompanying the force plays a Forced March Military Stratagem Marker. It then rolls again on the March table.

A force does not have to use the same type of March for the Force March as it did in its initial march (i.e., a force may Road March then Force March Cross Country).

A force which Force Marches may include units which were picked up along the way.

A result of “F” on the March table causes a force to lose one unit after completion of the march. Any unit that ever moved with the force is eligible for the loss, including those dropped off at any point.

Units which Force March may begin in a hex which contains enemy units, even if they moved into that hex during their initial movement.

\subsection{Interception}

During a player's movement segment, enemy units may enter a hex into which a moving enemy force entered by playing an Interception Military Stratagem. This is explained on the Stratagem Marker section of the Player Aid Card.

Note that there are movement limitations on units in a besieged city. See the Force Organization and Siege rules.

\subsection{Off-map Forces Box}

The off-map Forces area represents portions of the Roman Empire to the West. Units placed in the Off-Map Forces Box may enter the Strategic Map; units on the Strategic Map may enter the Off-Map Forces Box.

\subsubsection{Exiting the Off-Map Forces Box}

Units may exit the Off-Map Forces Box and enter the Strategic map. They may only do so by the Roman Supreme Leader playing a Political Stratagem marker. Off-Map reinforcements are placed as follows:

On any road hexagon on the West map edge (the hexagon may contain enemy units).
In any Roman controlled coastal city on the Mare Internum or Pontus Euxinus.
Off map reinforcements are placed on the entrance hexagon; there is no extra movement point cost for this placement. They may not move in the turn of entry. They may conduct combat normally.

\subsection{Entering the Off-Map Forces Box}

Roman units may enter the Off-Map Forces box simply by exiting the strategic map on any road hex on the Western map-edge using standard road movement and expending an additional movement point, or by completing a successful sea movement from any Roman coastal city on the Mare Internum (Mediterranean) or Pontus Euxinus (Black Sea).

Parthian units may never enter the Roman Off-map Forces box.

A unit may not both enter/exit the Off-Map Forces area in the same turn.

The Roman player may move units from the Strategic Map into the Off-Map Reinforcement Holding Area simply by conducting a successful March. It does not require a Political marker to move units into the Off-map Reinforcement Holding Area. However, once units are moved into the Off-Map Reinforcement Holding Area, they may not be moved out unless the Roman Supreme Leader plays a Political marker.

Any type of Roman unit may be moved into the Off-Map Holding Area, not just units which started as Off-Map Reinforcements.