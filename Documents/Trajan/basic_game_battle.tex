\section{BASIC GAME BATTLE}

When Battle occurs, temporarily remove the units involved from the map and place them on the Basic Game Battle Display (printed on the map) to resolve the battle (do NOT place them on the Battle Board—this is used in the Advanced Rules just place the units in a convenient location). See the Basic Game Battle Results Table, printed in the game charts.

\subsection{Battle Procedure}

\begin{enumerate}[1.]
  \setlength{\itemsep}{-0.7em}
  \item Determine tactical superiority
  \item Conduct the battles in three rounds:
 \begin{enumerate}[a.]
   \setlength{\itemsep}{-0.9em}
   \item Fire round
   \item Melee round
   \item Pursuit round
 \end{enumerate}
  \item During each round, Battle is conducted in the following order:
  \begin{enumerate}[a.]
    \setlength{\itemsep}{-0.9em}
    \item Player with tactical superiority attacks.
    \item Player with tactical inferiority attacks.
  \end{enumerate}
\end{enumerate}

\subsection{Tactical Superiority Determination}

If one leader plays a Military Stratagem marker and the other did not, then the leader who played the Military marker has Tactical Superiority.

If both players play a Military marker, or neither player does, then both players roll a dice, and add it to the leader factor of the leader with the highest leader factor in the battle; whoever has the higher die roll has tactical superiority. If there’s a tie, re-roll.

If one player pays more than one Military marker, then the player who played the higher number of Stratagem markers has Tactical superiority.

The player who does not have Tactical Superiority has Tactical Inferiority.

A battle consists of a maximum of the three rounds. A battle comes to an end at the end of three rounds, or if one side is completely eliminated.

\subsection{Conduct of Rounds}

The player with tactical superiority conducts his attack first, then the player with tactical inferiority. Note that combat is not simultaneous, and that the side with Tactical Superiority gets to inflict losses first before the other side gets to fight back.

If the player with tactical superiority completely eliminates the enemy force, the friendly force does not take any (further) losses.

Once a player has decided to attack with a force, he must use all the units in that force for combat; units may not be withheld from combat.

\subsection{Conduct of Combat}

There are three rounds to a battle. In certain rounds, different unit types have different advantages, representing differences in tactics and weaponry. To conduct combat, a player totals the combat strengths of the appropriate unit types, cross indexes the total on the Basic Game Combat Results Table, rolls a dice, and then applies the result.

\subsubsection{Fire Round}

Only troops capable of fire combat may fire. This includes the following unit types: Auxiliary Infantry, Horse Archers, and Civis units defending in a siege, or conducting a revolt.

While only these units can attack, all enemy units are subject to taking losses and making discipline checks.

\subsubsection{Melee Round}

In the melee round, all units may conduct combat. Certain unit types have their attack strengths increased.

All Heavy Infantry and Heavy Cavalry are doubled for melee combat. Elephants are tripled for melee combat. All other types use their printed strength.

\subsubsection{Pursuit Round}

In the pursuit round, all units may conduct combat. Certain unit types have their attack strengths increased.

Light Cavalry are doubled for pursuit combat. All other types use their printed strength.

\subsection{The Basic Game Combat\\*Results Table (CRT)}

The Basic Game CRT has the total combat strengths of units down the left side. This is cross indexed with the dice roll to give a result. The possible combat results are “No Effect”, “C”, 2, 5, 10, or 15.

If a numeric result is obtained, this is the number of enemy strength factors which are eliminated. That player may select which units of his own army he will have eliminated.

The enemy must lose at least that number of strength points. For example, if the enemy has only a 10 strength factor unit and takes a result of "5" then the entire unit is eliminated.

If a “C” result is obtained, the enemy player must then make a discipline check by rolling one die and checking the Discipline Table. Cross index the results with all discipline types. Only one roll is made per check, regardless of the number of friendly units. Note that this might result in certain units being disrupted and other units remaining in good order.

\subsection{Discipline check results}

The possible discipline results are as follows:

P - Pass/Maintain discipline: Units pass; they remain in good order.
F - Fail discipline: Units fail; they are flipped to their disrupted side.
G- Go Berserk: If this is the fire round, the units pass the discipline check. If this is the melee round, then the units are disrupted. If the pursuit round, the units are eliminated regardless if they are disrupted or good order.

\subsection{Effects of Disruption}

Disrupted units may not attack.
Units which are disrupted and become disrupted again are eliminated.
Disruption lasts only for the remainder of the battle.

Players can indicate disruption for units on the Basic Game Battle/Siege Displays by flipping them to their reverse side—there is no limited intelligence on the Battle/Siege Displays.

\subsection{Heavy Units}

All Heavy Infantry and Heavy Cavalry units use twice their printed defense strength for extracting losses, regardless of the round. For example, a Roman legion with a strength of 5 could be used to take 10 factors of losses.

\subsection{Maneuver Advantage}

Light Cavalry are not affected by losses on the Basic Game Combat Results Table in the Melee or Pursuit rounds if they have Maneuver Advantage. They still must make discipline checks, and are eliminated if disrupted twice. They are affected by losses in the Fire round.

In order to obtain Maneuver Advantage the player must have at least twice as many strength points of undisrupted light cavalry as the enemy has total undisrupted cavalry (of any type). For example, if a player has six factors of light armed horse, and the enemy has only three, then he has Maneuver Advantage.

As a player takes losses in battle, Maneuver Advantage may change from round to round.

Non-leader units with a strength of 0 strength units are still affected by Discipline Checks. They are eliminated by multiple disruptions, or when all other units in their force are eliminated. They may not attack by themselves.

\subsection{Leaders and Combat}

Each round, each leader may raise the discipline level of one friendly unit to his own discipline level. For example, Trajan could raise any one Roman unit to Imperator discipline. This must be declared prior to the discipline check.

If all friendly units are eliminated in combat, then all leaders are also eliminated. Otherwise, leaders are not affected by combat. They must retreat with a friendly retreating force.

Leaders are never disrupted by discipline checks. A unit the leader is applying its discipline value to would still be affected by adverse discipline results, however.

\subsection{Rally}

There is no "rallying" of disrupted units during battle itself. At the end of a battle, all disrupted units are considered rallied and flipped to their good order side.

\subsection{Effects of Terrain on Combat}

The terrain of the combat is determined by the hex in which the battle is taking place. The possible types of terrain and their effects are as follows:

Mountain
Neither side may take the Maneuver Advantage
No cavalry or elephant units have their offensive strength increased for any reason
Reduce all losses inflicted by 5, i.e. a 15 becomes a 10, 10 becomes 5, etc.

Swamp
Neither side may take the Maneuver Advantage
No cavalry or elephant units have their offensive strength increased for any reason.

River
Reduce all losses inflicted by 5.

City
Combat must be resolved as a siege if defender is in city. Otherwise, use other terrain.

\subsection{Example of Combat}

A Roman force of two 5 strength legions, two 4 strength Auxiliary Infantry, and three 2 strength equites (all Veterans) attacks a Parthian force of three 1 strength cataphracts (Veterans) and three 5 strength (Recruit) horse archers.

The Romans expend a Military marker and have tactical superiority. In the Fire Round, total Roman fire strength is 8 (for the two Auxiliary Infantry). The Romans roll a "4," which is No Effect. The Parthians then fire with a total fire strength of 15 (for the three horse archer units) and roll a "1," Discipline Check.

The Romans roll on the Discipline table and get a "5". Since all their units are Veterans, they pass the Discipline check.

In the Melee Round, the Romans attack with a total strength of 34 (20 for the two legions, as being heavy types, their combat strength is doubled, plus the value of the auxiliary infantry and cavalry).

The Romans roll a "4" which is a 5/C result. Since the Parthians have the Maneuver Advantage (twice or more cavalry strength) they take no losses (since all their units are cavalry). However, they are required to make a Discipline Check. They roll a "3". The Veteran cataphract units pass the check, but the Recruit Horse Archers fail and become disrupted.

In the Parthian Melee Round, they attack with a strength of 6 (the three cataphract units, doubled for being heavy types). They roll a "5” - no effect.

In the Pursuit Round, the Romans attack with a strength of 30 (as their cavalry is doubled). Note that the Parthians no longer receive the Maneuver Advantage owing to their horse archers being disrupted. The Romans roll a "1" calling for another Parthian Discipline check. The Parthians roll a "1" and all units fail. Since the Parthian horse archers were already disrupted, the second disruption eliminates them. The cataphracts are then disrupted. Since no undisrupted Parthian units remain, the Parthians cannot conduct combat in the Pursuit Round.

\subsection{Winning/Losing Battles}

A player is considered to have won a battle if the enemy has lost at least twice as many strength points as the friendly player, and this loss amounts to at least 25 percent of the enemy's combat strength or all enemy units, with the exception of leaders, are disrupted and all friendly units are undisrupted.

The battle is considered a draw if neither side meets these conditions, or both players meet at least one of the conditions.

\subsection{Effects of Winning, Losing\\*and Draws}

The loser of the battle must immediately retreat. To retreat, the winning player moves the entire losing force to any one adjacent hex. This hex can contain enemy units.

The retreating units may be attacked again this turn by an enemy force in the hex it retreated to if that enemy force has not yet conducted combat, and it is the enemy's turn, i.e. the loser must be the defender in order to be attacked again.

If there are other friendly units already in the hex retreated to, the retreating units may add their combat strength to them for any combat which is to be conducted in that hex later in the same combat segment.

The play of a Pursuit Military Stratagem allows the attacker to advance after combat and attack the retreating units.

The winner of the battle gets to pick additional Stratagem Markers, depending upon the size of the victory. Likewise, the loser of the battle loses Stratagem Markers, depending upon the size of the defeat.

\subsection{Size of Victory}

A victory may be either a Major Victory, Minor Victory, or Skirmish. This will depend upon the number of strength points eliminated in the loser's force.

Major Victory: Loser has 25 or more strength points eliminated.
Minor Victory: Loser has 5-24 strength points eliminated.
Skirmish: Loser has 0-4 strength points eliminated.

The total number of strength points lost by the winner is immaterial for determining the size of a victory.

In the case of a draw neither player retreats, and neither player gains or loses Stratagem markers.