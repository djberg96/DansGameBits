\section{LEADERS}

Leaders represent the major military or political figures who participated in the actual Parthian War. Leaders have two basic functions in the game: they are used to play Stratagem Markers, and to enhance the discipline of units they command.

\subsection{Leaders and Stratagem Markers}

Each leader may play a number of Stratagem Markers equal to his Leader rating each turn.

\subsection{Leaders and Discipline}

The discipline value of a leader may be used when making a Discipline Check for other units in the same hex (on the strategic map) or square (on the Advanced Game Battle Board). Each leader may change the discipline value of one other unit in its force to its own discipline value.

For example, the Roman Trajan leader unit, with his "Imperator" class discipline, could change the discipline of any one other Roman unit he is stacked with “I” class.

\subsection{Leaders and Units}

Leaders are treated exactly as other units, with the following exceptions:

Leaders are never affected by combat results, unless all other units are eliminated in their hex, at which point they are eliminated.
Eliminated leaders may never be replaced. Enemy leaders never eliminate each other if they are the only units in a hex; enemy leaders may freely enter hexes containing only enemy leaders.

Note that aside from the above, leaders function as normal units; i.e., they may enter hexes containing enemy units, block the movement of enemy units, etc.

\subsection{Supreme Leaders}

Each side has one Supreme Leader. The Romans have an Imperator. The Parthians have a King of Kings.

Trajan is the Roman Supreme Leader. If Trajan is eliminated for any reason, then the Roman player may designate any surviving Roman leader as Supreme leader if that leader is able to play the Imperator Stratagem Marker.

Chosroes is the Parthian Supreme Leader. If Chosroes is eliminated, then the Parthians may designate any surviving Core (blue) Parthian leader as Supreme leader, if the designated leader is stacked with the Parthian Court unit, and the designated leader plays the King of Kings Stratagem Marker.

Neither the Roman nor Parthian supreme leaders may be replaced the turn they are eliminated. See below for more details.

\subsection{Death of a Supreme Leader}

As noted in the Stratagem Marker rules below, each player receives a number of Stratagem Markers each turn equal to the leader value of his Supreme leader. If his Supreme Leader is killed, then he loses all his existing Stratagem Markers.

Since the loss of the Supreme leader means loss of all Stratagem Markers held by the player at the instant of the leader's death, then the Imperator or King of King markers can only be obtained on a subsequent turn by the player gaining more Stratagem Markers. That means the player will have to go out and win some battles, thus having his candidate for Supreme Leader prove his military prowess before assuming supreme power.

Additionally, if there is no Roman Supreme leader then the Romans may not receive off-map reinforcements (they are considered to be engaged in fighting for control of the Empire).
