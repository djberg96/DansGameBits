\section{REVOLTS}

Under certain circumstances, such as the play of a Revolt Political Action chit, a Civis unit may change sides, indicated by flipping it to the reverse side of its counter. This is called a Revolt.

Revolts occur when a leader plays a Revolt Political Stratagem marker against an enemy Civis unit in the same hex and rolls in the appropriate range. The Player Aid card details the range of dice rolls to cause a Revolt.

If there is a friendly leader in the city, then he may play a political marker to counter the enemy player's revolt marker.

\subsection{Revolt Combat}

If there are any enemy units in the city (i.e. the former garrison), this causes combat to be immediately conducted as follows:

The players conduct three rounds of combat. This is performed exactly as regular Battle (not Siege), with the following exceptions:

\begin{itemize}
  \item Only the Civis and garrison units participate in, and are affected by, the combat. Other units in the hex, friendly or enemy, are ignored.
  \item All units use their printed combat strength - there is no multiplication of strengths.
  \item There is no Maneuver Advantage.
\end{itemize}

If the combat results in the Civis being eliminated, then the garrison retains control of the city.

If the combat results in the Civis surviving, regardless of whether or not it is disrupted, then all garrison units (as well as any former relief force units) are considered to be outside the city and are now considered the besieging force. The player who initiated the revolt gets control of the city, and any formerly besieging units become a relief force.

If the only garrison unit in a city which has just revolted is a leader, then that leader is automatically eliminated.

Note this occurs immediately as the Revolt occurs. Combat may take place normally in the same hex during that turn's Combat Segment.