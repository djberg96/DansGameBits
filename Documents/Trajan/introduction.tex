\section{INTRODUCTION}

Trajan is a simulation of the Roman military campaign in the East (what is now Iraq/Iran/ Kuwait) in the years 114-117 AD. The Romans, under the leadership of the Emperor Marcus Ulpius Nerva Trajan, conquered the Parthian kingdom and extended the Roman Empire's boundaries to the Persian Gulf.

\subsection{General Course of Play}

Trajan is a two player game. One player controls the Romans, the other the Parthians. The objective of the Roman player is to gain control of critical cities within the Parthian Kingdom, while the Parthians must prevent this. Each side has a number of military units with which he can attack enemy forces and gain control of cities. Additionally, each side receives a designated number of Stratagem Markers each turn, representing various political, diplomatic, and military actions.

During each turn, players will draw Stratagem Markers, march their armies across the map, and engage in combat. Combat is conducted by transferring forces to the battle board, and resolving the combat in a quasi-tactical manner. Stratagem Markers are used to conduct intelligence operations, enhance combat, and most, importantly, to switch loyalties of certain game units.

Trajan's rules consist of two sections: The Basic Game gives a simple (but still historically valid) game and the Advanced Game, which gives a full historical simulation. A third section, Optional Rules, appeared in Moves Magazine \#66.

As much as possible, Trajan was designed to give a view of warfare from the perspective of ancient generals. Consequently, game components, including the map, units, and rules, are designed to reflect the realities of warfare in the 2nd Century AD.
