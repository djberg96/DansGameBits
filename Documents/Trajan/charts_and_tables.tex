\clearpage
\section{CHARTS AND TABLES}
\setlength{\arrayrulewidth}{0.6pt}

{
  \def\arraystretch{1.1}
  \resizebox{\textwidth}{!}{
    \begin{tabular}{|lll|}
      \hline & &\\[-2.0ex]
      \textbf{TERRAIN CHART} & &\\
      & \textbf{Movement Effects} & \textbf{Combat Effects (Basic Game)}\\
      City/Provincial Capital & Other Terrain & Siege defense if control\\
      \rowcolor{BabyBlue}Open & None & None\\
      Mountain & Must stop when marching Cross Country & No Manuever advantage;\\
      & & Cavalry/Elephants not increased offensively;\\
      & & Reduce losses by 5.\\
      \rowcolor{BabyBlue}Desert & None & None\\
      Swamp & Must stop when marching Cross Country & No Maneuver advantage;\\
      & & Cavalry/Elephants not increased offensively;\\
      \rowcolor{BabyBlue}River & Must stop when marching Cross Country & Reduce losses by 5\\
      \rowcolor{BabyBlue}& May use River Movement &\\
      Coast & Other terrain/Sea Movement & Other Terrain\\
      \rowcolor{BabyBlue}Sea & Can only move through & Not allowed\\
      \rowcolor{BabyBlue}& using Sea Movement &\\
      Roman Road & May use Road Movement & Other Terrain\\
      \rowcolor{BabyBlue}Trade Route & May use Road Movement & Other Terrain\\
      Lake & Same as River & Same as River\\
      \hline
    \end{tabular}
  }
}

{
  \def\arraystretch{1.1}
  \resizebox{\columnwidth}{!}{
    \begin{tabular}{|lcccccc|}
      \hline & & & & & &\\[-2.0ex]
      \textbf{DISCIPLINE TABLE} & & & & & &\\
      Discipline Class & \multicolumn{5}{c}{Die Roll} &\\
      & \textbf{1} & \textbf{2} & \textbf{3} & \textbf{4} & \textbf{5} & \textbf{6}\\
      Imperator & G & P & P & P & P & P\\
      \rowcolor{BabyBlue}Veteran & F & G & P & P & P & P\\
      Recruit & F & F & G & G & P & P\\
      \rowcolor{BabyBlue}Mob & F & F & F & F & G & P\\
      Barbarian & F & F & G & G & G & P\\
      & & & & & &\\
      \textbf{P}: Unit passes discipline check. & & & & & &\\
      \textbf{F}: Unit fails discipline check. & & & & & &\\
      \textbf{G}: Unit goes berserk. & & & & & &\\
      \hline
    \end{tabular}
  }
}

{
  \Huge
  \def\arraystretch{1.1}
  \resizebox{\textwidth}{!}{
    \begin{tabular}{|lccccc|}
      \hline & & & & &\\[-3.0ex]
      \multicolumn{6}{|c|}{\textbf{BASIC GAME COMBAT SYSTEM SUMMARY OF UNIT STRENGTHS}}\\
      \textbf{Unit Type} & \textbf{Fire Round} & \textbf{Melee Round} & \textbf{Pursuit Round} & \textbf{Maneuver} & \textbf{Defense}\\
      Heavy Infantry & x0 & x2 & x1 & No & x2\\
      \rowcolor{BabyBlue}Auxiliary Infantry & x1 & x1 & x1 & No & x1\\
      Heavy Cavalry & x0 & x2 & x1 & No & x2\\
      \rowcolor{BabyBlue}Equites Cavalry & x0 & x1 & x2 & Yes & x1\\
      Horse Archers & x1 & x1 & x2 & Yes & x1\\
      \rowcolor{BabyBlue}Elephants & x0 & x3 & x1 & No & x1\\
      Civis & x1 & x1 & x1 & No & x1\\
      \rowcolor{BabyBlue}Other Types & x0 & x0 & x0 & No & E\\
      & & & & &\\
      \textbf{x0:} Unit cannot attack (but defends normally). & & & & &\\
      \textbf{x1:} Unit uses its printed strength. & & & & &\\
      \textbf{x2:} Unit uses twice its printed strength when attacking. & & & & &\\
      \textbf{x3:} Unit uses three times its printed strength when attacking. & & & & &\\
      \textbf{Maneuver - Yes:} The unit type is used to determine maneuver combat. & & & & &\\
      \hspace{6.5em}\textbf{No:} The unit type is not used to determine maneuver combat. & & & & &\\
      \textbf{Defense - x1:} Use the unit's face value for extracting losses. & & & & &\\
      \hspace{5em}\textbf{x2:} The unit counts as twice as many strength factors when extracting losses. & & & & &\\
      \textbf{E:} Eliminated if all other friendly units are eliminated. & & & & &\\
      \hline
    \end{tabular}
  }
}

\clearpage

{
  \Large
    \def\arraystretch{1.1}
    \begin{tabularx}{\textwidth}{|@{}>{\hspace{0.5em}}l *7{>{\centering\arraybackslash}X}@{\hspace{0.5em}}|}
      \hline & & & & & & &\\[-2.5ex]
      \textbf{MARCH TABLE} & & & & & & & \\
      March Type & & \multicolumn{6}{c|}{Die Roll}\\
      & & \textbf{1} & \textbf{2} & \textbf{3} & \textbf{4} & \textbf{5} & \textbf{6}\\
      Road / Trade Route & & N & M/W & M2/A & M/F & M2/F & M2/W\\
      Cross Country & & N & M/S/W & M/S/A & M/A & M/F & M1/W\\
      River & & N & R/W & R/A & R/F & R/S & R/W\\
      Sea & & N & O/W & O/X & O/A & O/F & O/W\\
      \hline
    \end{tabularx}
}

{
  \def\arraystretch{1.1}
  \resizebox{\textwidth}{!}{
    \begin{tabular}{|lcccccc|}
      \hline & & & & & &\\[-2.0ex]
      \textbf{MARCH TABLE} & & & & & &\\
      March Type & \multicolumn{5}{c}{Die Roll} &\\
      & \textbf{1} & \textbf{2} & \textbf{3} & \textbf{4} & \textbf{5} & \textbf{6}\\
      Road / Trade Route & N & M/W & M2/A & M/F & M2/F & M2/W\\
      Cross Country & N & M/S/W & M/S/A & M/A & M/F & M1/W\\
      River & N & R/W & R/A & R/F & R/S & R/W\\
      Sea & N & O/W & O/X & O/A & O/F & O/W\\
      \hline
    \end{tabular}
  }
}

