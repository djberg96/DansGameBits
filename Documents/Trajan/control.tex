\section{CONTROL}

Control of cities is one of the key functions in the game.

\subsection{Definition of Control}

A player controls a city if he has a friendly Civis unit in the hex regardless of the presence of enemy units in the same hex. If the Civis unit has been eliminated, but a friendly unit is in the city itself, it is still considered controlled. If both players have friendly units in a city hex, then the player who has units inside the city controls it.

If none of the above conditions apply, then neither player controls the city. Note that these conditions must be maintained at all times for control to be in effect: i.e., if a player controls a city by having a friendly unit in it and then those units move out of it, he no longer controls the city. Simply being the last to pass through a city does not give control.

A city which is not occupied by either side's units is considered to be uncontrolled.

\subsection{Civis Units}

Civis units represent organic militia, bureaucracy, and political factions used to control a city. The units are backprinted, and, as the result of play of a political Stratagem marker can be converted from one side to the other. When on the red side, the unit represents Roman control When on the green side, Parthian control

There can only be one Civis unit in an individual city hex at any one time. There may never be a friendly and enemy Civis unit in the same hex at the same time.

Civis units are not inverted for reasons of limited intelligence or Disruption as are other units—Disruption of Civis units can be indicated by rotating the counter 180 degrees or by whatever means you prefer.

Civis units may never attack in Battle or Siege. They are not affected by combat results which are inflicted on friendly attacking units in the same hexagon. They attack only in a Revolt situation (see below). They defend normally and take losses while part of a defending force.

When in the Civis Holding Area, Civis units may be placed on either side; when replaced on the map, they are placed on the side of the player who is recruiting the unit, regardless of who controlled the Civis originally.

\subsection{Map Coding}

Note that cities are printed on the map as being either Roman, Parthian, Armenian and Arabian by a color or letter coding. This is distinct from control of a city. The coding on the map is used simply to indicate the general loyalties of the cities, not who controls them. Control is established and maintained only through occupation of the city itself by a player's units.
