\section{SUPPLY}

GENERAL RULE:

Units must be in Supply to use their full Combat Strength and Movement Allowance. If they are not in Supply, they are penalized with respect to Movement and Combat capabilities.

PROCEDURE:

Supply determination for movement purposes is made at the beginning of each Movement Phase. (Remember, the Germans have \textbf{two} Movement Phases). Thus, a unit in Supply at the beginning of a Movement Phase may use its full Movement Allowance. If a unit starts a Movement Phase unsupplied, and then moves into Supply during that Phase, it is still out of Supply for that Phase. Units are determined to be in Supply, for combat purposes, at the instant of combat; i.e., if an attacking unit had been in Supply at the beginning of the Combat Phase, but another, preceding combat had resulted in the Supply being cut, the attacking unit would \textbf{not} be Supplied for its own combat. To be in Supply, a unit must be able to trace a Supply Line to a Supply Source.

CASES:

\subsection{GERMAN SUPPLY}

\subsubsection{} To be in Supply, a German unit must be within \textbf{twenty} hexes of a Road hex (0120) on the \textbf{western} map edge. Neither the path to the Road nor the Road itself may be interrupted by Enemy units, Enemy-controlled hexes, or unpassable terrain. For purposes of tracing Supply, Friendly units \textbf{do} negate Enemy Zones of Control.

\subsubsection{} German units are also considered to be in Supply if they can trace a Line of Supply that is \textbf{twenty} hexes of a Road that eventually leads to the Road hex on the \textbf{western} map edge. These Movement Points are counted as per the type of unit (Infantry, Armor, etc.) seeking to be Supplied.

\subsubsection{} Although Roads do \textbf{not} cross Rivers for movement purposes (only Railroads do), they are considered to cross the Rivers for purposes of tracing Supply Lines.
\nobreak\\\\
\textcolor{blue}{This case applies only to rule 11.11 - connection to road that leads to hex 0120 - but does not apply to rule 11.12 - using the twenty movement points to trace supply directly to the eastern map edge. The cost to cross the river is paid regardless of the presence of roads in the latter case.}

\subsection{SOVIET SUPPLY}

\subsubsection{} To be in Supply, a Soviet combat unit must first trace a Line of Communications to \textbf{any} Leader (see 10.31). This Leader must then be able to trace a Line of Supply of any length to the \textbf{eastern} edge of the game map. The Leader unit thus coordinates Supply. The Leader unit thus coordinates Supply. Neither the Line of Communications nor the Line of Supply may be interrupted by (pass through hexes containing) Enemy units, Enemy Zones of Control, or unpassable terrain. For purposes of tracing Supply, Friendly units negate Enemy Zones of Control.

\subsubsection{} The length of the Line of Communications traced from the combat units to the Leader may be no more hexes than the rating of the Leader. Example: A Soviet division is four hexes from General Lukin; it is out of supply because Lukin's rating is three, and the maximum length of the Line of Communications from the combat unit to Lukin would have to be three. The unit may, of course, attempt to trace a Line of Communications to a different Leader.

\subsubsection{} Soviet Leaders may coordinate Supply for any number of Soviet combat units, but Soviet combat units must trace Supply through a Leader; they may not, in \textbf{any} case, trace a Line directly to the eastern edge of the game map.

\subsubsection{} Soviet Leaders are not automatically in Supply. They must trace a Line of Supply to the eastern edge of the game map to be considered in Supply. They do not need to trace a Line to another Leader.

\subsection{EFFECTS OF SUPPLY}

\subsubsection{} All units that are not in Supply have their Movement Allowance and Combat Strengths halved. All fractions are rounded down. Thus, an out-of-Supply 9-7 German Infantry Division is worth four Combat Strength Points and has a Movement Allowance of three.

\subsubsection{} A Unit's Strength or Movement Allowance may never be reduced below \textbf{one}.

\subsubsection{} All units are considered to be in Supply during the First Player-Turn (\textbf{not} Game-Turn) of their entrance into the game. After that, they must establish a Supply Line.

\subsubsection{} Units may remain out of Supply indefinitely; they are never lost through lack of Supply alone.

\subsubsection{} Supply never affects Zones of Control.
