\section{VICTORY CONDITIONS}

GENERAL RULE:

Victory is awarded to the Player according to the Victory Level Schedule. At the end of the game, the Players total the Points they have earned according to the Victory Point Schedule. This will result in a number of Victory Points attained by the Germans. Dependent on the total number of Victory Points amassed by the Germans, the Level of Victory and the Victor, may be determined.

PROCEDURE:

Each Player keeps track of his Victory Points on a separate sheet of paper. These Victory Points are awarded for attaining a variety of objectives as detailed in the Victory Point Schedule (Case 15.1). At the end of the game, the Points are totaled and the victory determined.

CASES:

\subsection{VICTORY POINT SCHEDULE}

\subsubsection{Points Awarded to the German Player}

The German Player receives Victory Points for occupying Soviet cities and certain hexes at the end of the game. Occupation means that the German Player must have a combat unit in \textbf{all} hexes of that city or be the last Player to have a unit pass through those hexes. In order to get credit for occupying a city, the German Player must be able to trace a line of hexes that is free of Enemy units or Enemy Zones of Control from the city to the \textbf{western} edge of the game map. Friendly units negate Enemy ZOC's for this purpose.

\textbf{German Victory Point Schedule:}

\begin{tabular}{l@{\hskip 1.2in}l}
  \textbf{City} & \textbf{Points for}\\
  & \textbf{Occupation}\\
  Vitebsk & 10\\
  Orsha & 5\\
  Mogilev & 5\\
  Smolensk & 25\\
  Roslovi & 10\\
  Yelnya & 10\\
  Rzhev & 5\\
  Gzhatsk & 5\\
  Vyazma & 15\\
  Kaluga & 20\\
  Hex 5907 & 20\\
  Hex 5915 & 20
\end{tabular}

In addition, the German Player receives Victory Points if the Soviet Player uses his South-Western Front Reinforcements; see Case 13.23.

\subsubsection{Points Awarded to\\*the Soviet Player}

\begin{enumerate}
  \item The Soviet Player receives \textbf{five} Victory Points for each \textbf{entire} German \textbf{division} eliminated - with the exception of the German Cavalry Division, for which the Soviet Player does not receive any Points. He receives no Points for individual \textbf{regts}.
  \item The Soviet Player receives \textbf{two} Victory Points for \textbf{each} city at the instant that he \textbf{recaptures} it from the Germans. In order to be recaptured, all hexes of the city had to have been held by the German Player at the end of the previous German Player-Turn. In addition, the German Player must have been able to trace a line of hexes as per Case 15.11 to the western edge of the game map. A city is considered recaptured if the Soviet Player then occupies \textbf{all} hexes of the city at the end of \textbf{his} Player-Turn. The Soviet Player \textbf{may} receive Points for recapturing a city \textbf{every time} it is recaptured.
\end{enumerate}

\subsection{LEVELS OF VICTORY}

The Soviet Player subtracts his Victory Point Total from the Point Total of the German Player. The total number of German Points is then compared with the following schedule and the winner, and the Level of Victory, is determined.

\begin{tabular}{ll}
  \textbf{German Victory Points} &\\
  (minus Soviet Points) & \textbf{Level of Victory}\\
  0 or fewer & Soviet Decisive\\
  1 to 25 & Soviet Strategic\\
  26 to 49 & Soviet Marginal\\
  50 to 79 & German Marginal\\
  80 to 124 & German Strategic\\
  125 or more & German Decisive
\end{tabular}