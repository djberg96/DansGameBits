\section{COMBAT}

GENERAL RULE:

Combat occurs between adjacent opposing units at the Phasing Player's discretion. The Phasing Player is the attacker, the non-Phasing player is the defender, regardless of the overall strategic situation.

PROCEDURE:

Total the Attack Strengths of all the attacking units involved in a specific attack and compare it to the total Defense Strength of the units in the hex under attack. State the comparison as a probability ratio: Attacker's Strength to Defender's Strength. Round off the ratio in favor of the defender to conform to the simplified odds found on the Combat Results Table; roll the die and read the result on the appropriate line under the odds. Apply the result immediately, before resolving any other attacks being made during the Combat Phase.

CASES:

\subsection{WHICH UNITS MAY ATTACK}

\subsubsection{} Units may attack only during their own Friendly Combat Phase (see also Overrun, 6.5). They may then attack any and all Enemy units which are adjacent to them. Only those units directly adjacent to a given Enemy unit may participate in an attack upon that unit.

\subsubsection{} Attacking is completely voluntary; units are never compelled to attack, and not every unit adjacent to an Enemy unit need participate in any attack. A Friendly unit in a stack that is not participating in a given attack is never affected by the results of that attack.

\subsubsection{} An Enemy-occupied hex may be attacked by as many units as can be brought to bear in the six adjacent hexes.

\subsubsection{} No unit may attack more than once per Combat Phase and no Enemy unit may be attacked more than once per Combat Phase. [Remember, Overrun is \textbf{not} combat].

\subsection{MULTIPLE UNIT AND\\MULTI-HEX COMBAT}

\subsubsection{} All units in a given hex must be attacked as a single Defense Strength. The defender may not withhold a unit in a hex under attack. Different units in a hex may not be attacked separately, nor may one unit be attacked without involving the other units in the same combat.

\subsubsection{} Other units in a hex that contains an attacking unit need not participate in that same combat or any other attack. Thus when one unit in a stack is attacking a given hex, the other units in the stack could attack a different hex, or not attack at all.

\subsubsection{} If a unit(s) is adjacent to more than one Enemy-occupied hex, it could attack all of them in a single combat. Thus, units in a single hex may attack more than one hex. The only requirement is that all attacking units must be adjacent to all defending units.

\subsubsection{} A given unit's Attack and/or Defense Strength is always unitary; that is, it may not be divided among different combats either for attack or defense.

\begin{flushleft}
  \subsection{EFFECTS OF TERRAIN ON COMBAT}
\end{flushleft}

\subsubsection{} Units defending in certain types of terrain may have their Defense Strength increased. This is always expressed as a multiple of the Defense Strength. (A unit with a Defense Strength of four in a Major City defends with a Strength of eight).

\subsubsection{}] A defending unit may obtain the doubling effect of Rivers only if \textbf{all} the attacking units are attacking across River hexsides. If one or more attacking units is not attacking across a River hexside, then the defender does not obtain any defensive advantage from the River.

\subsubsection{} The effects of terrain are cumulative. You always add the multiples together and then subtract "one" to obtain your new multiple. That is, if a unit is in Forest terrain and is being attacked across a River, then its Defense Strength is tripled (2 + 2 - 1 = 3).

\subsubsection{} See the Terrain Effects Chart (6.7) for a complete list of effects of terrain.

\subsection{COMBAT RESOLUTION}

Combat odds are always rounded off in favor of the defender. For example, an attack with a combined Attack Strength of 26 against a hex defending with a Defense Strength of 9 (26 to 9) would round off to the next lowest odds column on the CRT, "2 to 1". That column would be used for resolving the attack.

\subsection{COMBAT RESULTS TABLE} (See page R8).

\begin{flushleft}
  \subsection{EXPLANATION OF COMBAT RESULTS}
\end{flushleft}

\subsubsection{} German units have a number of Strength Levels. They may be reduced in Strength as a result of combat by one Level at a time; i.e., if a German unit takes a one-step combat loss, the unit's Marker is replaced by the next lowest Strength Level Marker for that unit. If there are no more steps left in the unit, the unit is eliminated.

\subsubsection{} All Soviet units consist of one step \textbf{only}. Therefore, if a Soviet unit receives a one step loss, it is eliminated. Each Soviet unit is considered to be one step.

\subsubsection{} All German Panzer, Motorized, Mech and Cavalry units have \textbf{two} steps, the second step being on the reverse side of the Marker with the original Strength. All German Infantry Divisions have four steps (9-7, 4-7, 2-7, and 1-7) and can be reduced accordingly.

\subsubsection{} All combat results are expressed in terms of steps or hexes retreated. The letters "A" and "D" stand for Attacker and Defender, respectively. A result of "Ae" and "De" means that \textbf{all} steps for the units involved are lost and no retreat option is possible.

\subsubsection{} A result of A or D plus a number (e.g., A1, D2, etc.) means that the affected unit(s) must \textbf{either} lose the given number of steps \textbf{or} retreat \textbf{all} units in that combat the given number of hexes. The Player whose units are so affected may not take a step loss \textbf{and} retreat; he must either retreat \textbf{or} take step losses.

When a loss of one step (or more) is required or chosen, this does \textbf{not} mean one step is removed from \textbf{each} affected unit. It means that the defender (or attacker) removes one step from any one unit involved. Example: If three Soviet units are defending against a German attack and the CRT shows a result of "D1", the Soviet Player has the option of either removing \textbf{one} of his units (thus eliminating the one step) and, leaving the remaining units in place, \textbf{or} retreating \textbf{all three} units one hex.

\subsubsection{} Some results on the CRT are "split" results; e.g., "D1/A1". In a split result, the \textbf{defender} always takes his result first, whether it is a step loss or a hex retreat. Then the attacker takes his result. If any \textbf{attacking} units remain in the \textbf{original} hex, they may advance after combat, provided that the defending hex has been vacated. The defender may never advance in a split result. A split result halts an Overrun (see Case 6.52).

\subsubsection{} A result of "Engaged" means that both sides \textbf{must} lose one step each; no retreat option is available. In addition, neither side may advance after combat. If this result occurs during an Overrun, the Overrunning units remain in the hex from which the Overrun originated; the Engaged result is applied, and neither unit may move further.

\subsection{RETREATS}

\subsubsection{} Retreats are always optional. The Player may choose to lose steps rather than retreat (see Case 9.64). However, a unit may never retreat into or through an \textbf{Enemy} unit or an \textbf{Enemy} Zone of Control, unless the hex is in an Enemy ZOC and occupied by a Friendly unit. Units may not retreat through impassable Lake hexsides.

\subsubsection{} Retreats of Friendly units are conducted by the Enemy Player, even in the case of split results; i.e., the Path of Retreat is always determined by the opposing Player, within the guidelines of Case 9.73.

\subsubsection{} A retreating unit must, if possible, retreat into a vacant hex. If no vacant hex is available, it may retreat into or through a hex that is occupied by a Friendly unit. Units may \textbf{not} retreat into a hex in violation of stacking restrictions; units that are forced to do so are eliminated instead. Thus, if the German units are forced to retreat into \textbf{or through} a hex occupied by two other German units, only one of the retreating units may successfully retreat; the other is eliminated because it is in violation of stacking restrictions.

\subsubsection{} Units may retreat \textbf{through} other Friendly units, within the bounds of Case 9.73, without disturbing the non-retreating units. The non-retreating units are not affected by the retreating units; they do not have to move out of the way of the retreating units.

\subsubsection{} If a unit is forced to retreat into a Friendly occupied hex and that hex then undergoes an attack (regular or Overrun) the \textbf{retreated} unit does not add its Strength to the units in the hex. However, if that new hex suffers \textbf{any} combat results (loss or retreat), the previously retreated unit is \textbf{automatically} eliminated, regardless of whether the Player decides to retreat or not.

\subsection{ADVANCE AFTER COMBAT}

\subsubsection{} Whenever an Enemy unit is forced to retreat (or is eliminated) leaving the hex vacant as a result of combat, it will leave a path of vacant hexes behind it called the Path of Retreat. Any or all Friendly victorious units which participated in the combat are allowed to advance along the Enemy Path of Retreat.

\subsubsection{} The advancing victorious units may cease advancing in any hex along the Path of Retreat.

\subsubsection{} Advancing victorious units may \textbf{ignore} Enemy Zones of Control.

\subsubsection{} An advancing unit may not stray from the Path of Retreat (however, see Case 9.86).

\subsubsection{} The option to advance must be exercised immediately, before any other combat resolution. Units are never forced to advance after combat (but see Cases 6.51 and 6.57). After advancing, units may neither attack (nor be attacked if they are advancing defending units) in that Phase (see Case 9.14), even if their advance places them adjacent to Enemy units whose battles are yet to be resolved or who were not involved in combat. However, advances are useful in cutting off the retreat of Enemy units whose combat has not yet been resolved.

\subsubsection{} If \textbf{all} units in a hex are eliminated, the victorious units may advance a maximum of two hexes after combat. The first hex must be the hex formerly occupied by the destroyed unit(s); the second hex may be any empty hex.

\subsubsection{} Any victorious unit may advance after combat, regardless of whether it was the attacker or the defender in the battle fought (however, see Case 9.86).

\subsubsection{} Advance after combat does not apply to Overrun, which is part of movement.
