\section{COMBAT}

GENERAL RULE:

Combat occurs between adjacent opposing units at the Phasing Player's discretion. The Phasing Player is the attacker, the non-Phasing player is the defender, regardless of the overall strategic situation.

PROCEDURE:

Total the Attack Strengths of all the attacking units involved in a specific attack and compare it to the total Defense Strength of the units in the hex under attack. State the comparison as a probability ratio: Attacker's Strength to Defender's Strength. Round off the ratio in favor of the defender to conform to the simplified odds found on the Combat Results Table; roll the die and rea the result on the appropriate line under the odds. Apply the result immediately, before resolving any other attacks being made during the Combat Phase.

CASES:

\subsection{WHICH UNITS MAY ATTACK}

\subsubsection{} Units may attack only during their own Friendly Combat Phase (see also Overrun, 6.5). They may then attack any and all Enemy units which are adjacent to them. Only those units directly adjacent to a given Enemy unit may participate in an attack upon that unit.

\subsubsection{} Attacking is completely voluntary; units are never compelled to attack, and not every unit adjacent to an Enemy unit need participate in any attack. A Friendly unit in a stack that is not participating in a given attack is never affected by the results of that attack.

\subsubsection{} An Enemy-occupied hex may be attacked by as many units as can be brought to bear in the six adjacent hexes.

\subsubsection{} No unit may attack more than once per Combat Phase and no Enemy unit may be attacked more than once per Combat Phase. [Remember, Overrun is \textbf{not} combat].

\subsection{MULTIPLE UNIT AND MULTI-HEX COMBAT}

\subsubsection{} All units in a given hex must be attacked as a single Defense Strength. The defender may not withhold a unit in a hex under attack. Different units in a hex may not be attacked separately, nor may one unit be attacked without involving the other units in the same combat.

\subsubsection{} Other units in a hex that contains an attacking unit need not participate in that same combat or any other attack. Thus when one unit in a stack is attacking a given hex, the other units in the stack could attack a different hex, or not attack at all.

\subsubsection{} If a unit(s) is adjacent to more than one Enemy-occupied hex, it could attack all of them in a single combat. Thus, units in a single hex may attack more than one hex. The only requirement is that all attacking units must be adjacent to all defending units.

\subsubsection{} A given unit's Attack and/or Defense Strength is always unitary; that is, it may not be divided among different combats eitehr for attack or defense.

\section{EFFECTS OF TERRAIN ON COMBAT}

\subsubsection{} Units defending in certain types of terrain may have their Defense Strength increased. This is always expressed as a multiple of the Defense Strength. (A unit with a Defense Strength of four in a Major City defends with a Strength of eight).

\subsubsection{}] A defending unit may obtain the doubling effect of Rivers only if \textbf{all} the attacking units are attacking across River hexsides. If one or more attacking units is not attacking across a River hexside, then the defender does not obtain any defensive advantage from the River.

\subsubsection{} The effects of terrain' are cumulative. You always add the multiples together and then subtract "one" to obtain your new multiple. That is, if a unit is in Forest terrain and is being attacked across a River, then its Defense Strength is tripled (2 + 2 - 1 = 3).

\subsubsection{} See the Terrain Effects Chart (6.7) for a complete list of effects of terrain.

\subsection{COMBAT RESULTS TABLE} (See page R8).

\subsection{EXPLANATION OF COMBAT RESULTS}

\subsubsection{} German units have a number of Strength Levels. They may be reduced in Strength as a result of combat by one Level at a time; i.e., if a German unit takes a one-step combat loss, the unit's Marker is replaced by the next lowest Strength Level Marker for that unit. If there are no more steps left in the unit, the unit is eliminated.

\subsubsection{} All Soviet units cosist of one step \textbf{only}. Therefore, if a Soviet unit receives a one step loss, it is eliminated. Each Soviet unit is considered to be one step.

\subsubsection{} All German Panzer, Motorized, Mech and Cavalry units have \textbf{two} steps, the second step being on the reverse side of the Marker with the original Strength. All German Infantry Divisions have four steps (9-7, 4-7, 2-7, and 1-7) and can be reduced accordingly.

\subsubsection{} All combat results are expressed in terms of steps or hexes retreated. The letters "A" and "D" stand for Attacker and Defender, respectively. A result of "Ae" and "De" means that \textbf{all} steps for the units involved are lost and no retreat option is possible.

\subsubsection{} A result of A or D plus a number (e.g., A1, D2, etc.) means that the affected unit(s) must \textbf{either} lose the given number of steps \textbf{or} retreat \textbf{all} units in that combat the given number of hexes. The Player whose units are so affected may not take a step loss \textbf{and} retreat; he must either retreat \textbf{or} take step losses.