\section{SOVIET LEADERS}

GENERAL RULE:

The Soviet Leader counters represent organizations of logistical and support torops, at the army level, headed by the general named on the counter. Soviet Leaders have a dual existence: they are treated as regular combat units - although they have no attack capabilities - and they coordinate and aid Soviet supply and combat.

CASES:

\subsection{THE LEADERSHIP RATING}

The rating of each individual Leader represents three different capabilities: a) the Radius, in the number of \textbf{hexes}, within which Soviet combat units must be to be considered in supply for both movement and combat (see Case 11.2) and within which a Soviet unit must be to initiate an attack; b) the maximum number of Combat Points that the Leader may possibly add to units with which it is directly stacked; and c) the \textbf{Defensive} Combat Strength of the Leader.

\subsection{CHARACTERISTICS\\OF SOVIET LEADERS}

\subsubsection{} Leaders have a Movement Allowance of \textbf{ten}. However, although they move like Motorized units on the roads, they move like Infantry in the Forests. Soviet Leaders may move on Railroads and do not count against the eight-unit limit; they \textbf{may} move on Railroads by themselves. They do not need to be stacked with a regular combat unit at any time.

\subsubsection{} Soviet Leaders exert a Zone of Control. They are treated in all ways like normal combat units, except that they have no Attack Strength.

\subsubsection{} Soviet Leaders may only enter into an Enemy Zone of Control if accompanied by at least one regular combat unit that has an attack capability. If a Leader advances into an Enemy ZOC with an Untried unit, and that unit is revealed to be a 0-0-6, 0-10 or 0-1-6, then the Leader must immediately retreat to the thex from which it entered the Enemy ZOC. The Enemy unit may not move after this "retreat". If the Leader cannot, for any reason, retreat, it is eliminated.

\subsubsection{} Each Soviet Leader is considered to be one full step for purposes of combat losses. If a Soviet Leader is in a stack of units suffering a combat loss, the Soviet Leader may be eliminated to satisfy the step loss, if so desired.

\subsubsection{} Leaders may stack with other units (see Section 7.0). Soviet Leaders are not required to stack regular combat units.

\subsubsection{} If a Leaders is in a stack of units that suffers a combat result, the Leader undergoes all retreats undertaken or may be used to absorb step losses, if so desired. It may also advance after combat with regular combat units. If the combat units in the stack divide their attack (9.23), the Leader must be assigned to \textbf{one} of the attacks, suffering any results that that attack incurs.

\begin{flushleft}
  \subsection{CAPABILITIES OF SOVIET LEADERS}
\end{flushleft}

\subsubsection{} Soviet Leaders coordinate supply for \textbf{all} Soviet combat units. combat units must be able to trace a Line of Communications, in hexes equal to the Radius rating of a Leader who, in turn, must be able to trace a Line of Supply to the eastern edge of the game map, in order for those units to be in supply (see Case 11.2). Soviet combat units are not in supply unless they can first trace a Line of Communications to a Leader, even if they are adjacent to the eastern edge of the game map!

\subsubsection{} Soviet Leaders may coordinate supply for any number of combat units. There is no limit to the number of combat units. There is no limit to the number of units that may be supplied through any one Leader. "Army" designations have no effect on Leadership.

\subsubsection{} Combat units may use the Railroad Movement Bonus without the presence of a Leader to coordinate supply; i.e., units may move thirty hexes on a Railroad without a Leader or supply.

\subsubsection{} Soviet Leaders, themselves, are in supply simply by tracing a Line of Supply of any length to the eastern edge of the game map (see Section 11.0).

\subsubsection{} Soviet combat units may not attack unless they are within the Leadership Radius of a Soviet Leader. Thus, a unit may be out of \textbf{supply} because the Leader may not be able to trace a Line of Supply to the eastern edge, yet still be able to attack - albeit at half-strength - if it can trace a Line of Communications to a Leader (see Case 11.21). Combat units may defend at half Strength without the presence of a Leader; they do not need a Leader to defend.

\subsubsection{} Soviet combat units stacked \textbf{directly} with a Leader may \textbf{add} the Leadership Value of the Leader to an \textbf{attack}, not to a defense, but the value added to the combat units' total Strength may never be more than equal to its given Strength after any deductions for supply, etc. Soviet Leader Points may not be split for these purposes. Example: A 6-8-6 unit is stacked with a Leader with a value of "four" and in supply; if it attacks, it attacks with a total strength of "10". (six plus four for the Leader). If it was out of supply, it would attack with a strength of "six" (3+3). If a 1-2-6 si stacked, in supply, with a "4" Leader, its total Attack Strength is "2" as it may never add more than its own total value (1 + 1 = 2).


\subsection{EVACUATING SOVIET LEADERS\\*{[}Optional Rule{]}}

The Soviet Player may \textbf{evacuate} a maximum of one Soviet Leader per Game-Turn, if necessary. He may do this only twice in the game. If, at the beginning of the Soviet Movement Phase, there is a Soviet Leader who is completely surrounded by Enemy units and/or Enemy ZOC's (ZOC's are \textbf{not} considered negated by Friendly units for this purpose) he may be "evacuated". To do this, the Soviet Player simply removes him from the hex and takes him off the game map. The evacuated Leader may return to the game the following Turn, or later, through Entrance Hex "X". Only one Soviet Leader may be evacuated per Game-Turn, and only \textbf{two} per game, and it must be done at the \textbf{beginning} of the Soviet Movement Phase. This is an Optional Rule, its use will tend to tip the balance of the game in the direction (if not the favor) of the Soviets a bit.

