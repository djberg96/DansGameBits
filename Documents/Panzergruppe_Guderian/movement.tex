\section{MOVEMENT}

GENERAL RULE:

During the Movement Phase, the Phasing Player may move as many or as few of his units as he wishes. During his Friendly Movement Phase (and for mechanized units, again during the German Mechanized Movement Phase), each unit may move as many or as few hexes as desired as long as its Movement Allowance is not exceeded in a \textbf{single} Movement Phase. Unused Movement Points may not be accumulated or transferred.

PROCEDURE:

Move each unit individually, tracing the path of its movement through the hexagonal grid. Once the Player's hand is removed from the unit, movement is considered completed.

CASES:

\subsection{HOW TO MOVE UNITS}

\subsubsection{} During a Movement Phase, all, some or none of a Player's units may be moved. No other units may be moved. Combat may not occur in this Phase; however, \textbf{Overrun} - a form of combined combat and movement - may take place in this Phase. See Case 6.5.

\subsubsection{} Movement is calculated in terms of Movement Points. Basically, each unit expends one Movement Point of its total Movement Allowance for each Clear terrain hex it enters; other terrain costs \textbf{more} than one Movement Point to enter or cross. These effects are summarized on the Terrain Effects Chart (6.7).

\subsection{MOVEMENT INHIBITIONS AND PROHIBITIONS}

\subsubsection{} A Friendly unit may never enter a hex containing an Enemy unit.

\subsubsection{} A unit must stop upon entering an Enemy-controlled hex (see Section 8.0); Once a unit is in an Enemy Zone of Control, it may not leave that hex voluntarily.

\subsubsection{} A unit may not expend more Movement Points that its total Movement Allowance in any \textbf{one} Movement Phase. (Note that German mechanized units have \textbf{two} Movement Phases and may expend their full Movement Allowance in \textbf{each} phase). A unit may use all, some or none of its Movement Points in a given Phase. However, a unit may not "save" Movement Points for another Turn, nor may unused Points be transferred to another unit.

\subsubsection{} Units may move only during their Friendly movement Phase(s), although there may be some movement as a result of combat, in terms of advances and retreats. These are not considered to be "movement" and do not require the expenditure of Movement Points.

\subsubsection{} Units that are out of supply (Section 11.0) have their Movement Allowance halved, dropping all fractions.

\subsection{RAIL MOVEMENT}

\subsubsection{} The Soviet Player - and only the Soviet Player - may move up to eight combat units, plus any number of Leaders, by Rail each Turn. For the purposes of Rail Movement, each Soviet division tank or mechanized counts as \textbf{three} combat units.

\subsubsection{} In order to move by Rail, a unit must start the Movement Phase \textbf{in a} Railroad hex and it must finish that Phase on a Railroad hex. It must move from Rail hex to adjacent, connected Rail hex only in that Phase. The units may not enter an Enemy Zone of Control when travelling by Rail. Units entering the game as reinforcements may use Rail Movement if they enter the game map on a Railroad hex. Reinforcements using Rail Movement \textbf{do} count against the eight combat unit maximum. (Remember, Leaders do not count).

\subsubsection{} Soviet units using Rail Movement may move \textbf{thirty} hexes by Rail in any one Turn. German Air Interdiction may limit movement by Rail by increasing the cost per Rail hex (see Section 12.0). Terrain has no effect on Rail Movement.

\subsubsection{} Units do not have to be in supply to use the \textbf{full} Railroad Movement bonus, nor do they have to be within a Leader's radius (see Case 10.21).

\subsubsection{} Aside from providing the Soviets with added movement capabilities, the Railroads have no other function in the game. They affect neither supply nor normal movement.

\subsubsection{} German units may not use Rail Movement.

\subsection{SPECIAL SOVIET MOVEMENT RESTRICTIONS}

For the first \textbf{six} Game-Turns, no Soviet unit (with the exception of the units noted in Case 5.23) may enter, voluntarily or involuntarily, the first two hexrows of the western edge of the game map (0100 and 0200). Soviet units may \textbf{attack} German units that are in these hexrows, but they may not advance after combat. Soviet Units may not attempt to \textbf{Overrun} German units in these hexrows. If forced to retreat into one of these hexrows, a Soviet unit is eliminated. However, Soviet units may trace Supply Lines and Leadership Radii through these hexrows.

\subsection{OVERRUN}

