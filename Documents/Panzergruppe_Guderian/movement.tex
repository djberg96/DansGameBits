\section{MOVEMENT}

GENERAL RULE:

During the Movement Phase, the Phasing Player may move as many or as few of his units as he wishes. During his Friendly Movement Phase (and for mechanized units, again during the German Mechanized Movement Phase), each unit may move as many or as few hexes as desired as long as its Movement Allowance is not exceeded in a \textbf{single} Movement Phase. Unused Movement Points may not be accumulated or transferred.

PROCEDURE:

Move each unit individually, tracing the path of its movement through the hexagonal grid. Once the Player's hand is removed from the unit, movement is considered completed.

CASES:

\subsection{HOW TO MOVE UNITS}

\subsubsection{} During a Movement Phase, all, some or none of a Player's units may be moved. No other units may be moved. Combat may not occur in this Phase; however, \textbf{Overrun} - a form of combined combat and movement - may take place in this Phase. See Case 6.5.

\subsubsection{} Movement is calculated in terms of Movement Points. Basically, each unit expends one Movement Point of its total Movement Allowance for each Clear terrain hex it enters; other terrain costs \textbf{more} than one Movement Point to enter or cross. These effects are summarized on the Terrain Effects Chart (6.7).

\subsection{MOVEMENT INHIBITIONS AND PROHIBITIONS}

\subsubsection{} A Friendly unit may never enter a hex containing an Enemy unit.

\subsubsection{} A unit must stop upon entering an Enemy-controlled hex (see Section 8.0); Once a unit is in an Enemy Zone of Control, it may not leave that hex voluntarily.

\subsubsection{} A unit may not expend more Movement Points that its total Movement Allowance in any \textbf{one} Movement Phase. (Note that German mechanized units have \textbf{two} Movement Phases and may expend their full Movement Allowance in \textbf{each} phase). A unit may use all, some or none of its Movement Points in a given Phase. However, a unit may not "save" Movement Points for another Turn, nor may unused Points be transferred to another unit.

\subsubsection{} Units may move only during their Friendly movement Phase(s), although there may be some movement as a result of combat, in terms of advances and retreats. These are not considered to be "movement" and do not require the expenditure of Movement Points.

\subsubsection{} Units that are out of supply (Section 11.0) have their Movement Allowance halved, dropping all fractions.

\subsection{RAIL MOVEMENT}

\subsubsection{} The Soviet Player - and only the Soviet Player - may move up to eight combat units, plus any number of Leaders, by Rail each Turn. For the purposes of Rail Movement, each Soviet division tank or mechanized counts as \textbf{three} combat units.

\subsubsection{} In order to move by Rail, a unit must start the Movement Phase \textbf{in a} Railroad hex and it must finish that Phase on a Railroad hex. It must move from Rail hex to adjacent, connected Rail hex only in that Phase. The units may not enter an Enemy Zone of Control when travelling by Rail. Units entering the game as reinforcements may use Rail Movement if they enter the game map on a Railroad hex. Reinforcements using Rail Movement \textbf{do} count against the eight combat unit maximum. (Remember, Leaders do not count).

\subsubsection{} Soviet units using Rail Movement may move \textbf{thirty} hexes by Rail in any one Turn. German Air Interdiction may limit movement by Rail by increasing the cost per Rail hex (see Section 12.0). Terrain has no effect on Rail Movement.

\subsubsection{} Units do not have to be in supply to use the \textbf{full} Railroad Movement bonus, nor do they have to be within a Leader's radius (see Case 10.21).

\subsubsection{} Aside from providing the Soviets with added movement capabilities, the Railroads have no other function in the game. They affect neither supply nor normal movement.

\subsubsection{} German units may not use Rail Movement.

\subsection{SPECIAL SOVIET MOVEMENT RESTRICTIONS}

For the first \textbf{six} Game-Turns, no Soviet unit (with the exception of the units noted in Case 5.23) may enter, voluntarily or involuntarily, the first two hexrows of the western edge of the game map (0100 and 0200). Soviet units may \textbf{attack} German units that are in these hexrows, but they may not advance after combat. Soviet Units may not attempt to \textbf{Overrun} German units in these hexrows. If forced to retreat into one of these hexrows, a Soviet unit is eliminated. However, Soviet units may trace Supply Lines and Leadership Radii through these hexrows.

\subsection{OVERRUN}

During a Movement Phase - and \textbf{only} during a Movement Phase - the Phasing Player may attempt to Overrun any Enemy unit. For game purposes, Overrun is considered to be a function of movement. Both sides may conduct Overruns in their respective Movement Phases; the Germans, having two Movement Phases, may conduct Overruns in both Movement Phases.

\subsubsection{} To conduct an Overrun, the Phasing Player, in a Friendly Movement Phase, moves his unit(s) adjacent to the target hex. All of the units which participate in the Overrun must be in the same hex. The Phasing units, if they wish to Overrun, must then expend three additional Movement Points to attack \textbf{all} Enemy units in the target hex. The Phasing units' Attack Strengths are reduced by half, dropping fractions. All terrain rules and Supply rules are in effect. If, as a result of the Overrun, the Enemy hex is vacated, the Friendly, Overrunning Player \textbf{must} then move all of the attacking units into the vacated hex (however, see Case 6.52). There is no cost for moving into the vacated hex. The victorious Overrunning units may then if they so desire, continue normal movement if they have any remaining Movement Points and they are not \textbf{then} in an Enemy Zone of Control. The Phasing units may conduct further Overruns if they have the necessary Movement Points.

\subsubsection{} If an Overrun attack fails to dislodge the Enemy units from the hex, the Overrunning units may not move any further in that Movement Phase. Overrunning units do \textbf{not} move into a vacated target hex if the combat result is a "split" result (see Case 9.67) or an "engaged" result (see Case 9.68). In addition, a "split" or "engaged" result stops movement for the affected units for the remainder of that Phase. Any combat result where the attacker/Overrunning unit must take a loss or retreat halts movement for that Phase.

\subsubsection{} Soviet Leaders may be Overrun like any other units. However, Soviet Leaders may \textbf{not} conduct Overruns by themselves as they have no Attack Strength (see Cases 10.22 and 10.36).

\subsubsection{} All units conducting an Overrun against an individual target hex \textbf{must} start the Movement Phase in the same hex. Individual Overruns may be conducted against more than one unit, but all these units must be in the same hex. Individual Overruns may be conducted against more than one unit, but all these units must be in the same hex. You may not attempt to Overrun more than one hex at any one time, although you may conceivably conduct more than one Overrun in a given Movement Phase. A defending unit may conceivably be Overrun by different unit more than once in a Phase.

\subsubsection{} A unit conducting an Overrun attacks at \textbf{half} Strength. The halving is done after all adjustments to the Combat Strength have been made. All fractions remaining after this final adjustment are then dropped. Thus, a German Panzer Division (worth eight Combat Strength Points at face value) that is out of supply and conducting an Overrun would Overrun with a Strength of four. (Its total Strength would be doubled for divisional integration, see Case 7.3, halved for being out of supply and then halved again for Overrun). Note that Soviet units that begin a Friendly Movement Phase beyond the Radius of a Leader may \textbf{not} conduct Overruns.

\subsubsection{} When conducting an Overrun attack, Players may ignore the Zones of Control of units which are exterting influence on the hex from which the Overrun attack is coming. That is, they may occupy the target hex if the Overrun is successful regardless of other Enemy ZOC's. However, if the Friendly unit finishes an Overrun in an Enemy Zone of Control, it may not \textbf{move} any further in that Movement Phase; however, it may conduct another Overrun on an adjacent hex providing it has the necessary Movement Points.

\subsubsection{} Units conducting Overruns may \textbf{not} conduct "advance after combat" (see Case 9.7); they can move into the vacated target hex, but may not move further without expending Movement Points. Remember, Overrun is movement, not combat.

\subsubsection{} Enemy units that \textbf{defend} successfully against an Overrun attack to not move into a vacated hex or advance after combat. They remain in their hex.

\subsection{DISRUPTION}

\subsubsection{} Units that are \textbf{defending} against an Overrun and suffer any loss or retreat (not including an Engaged) as a result of the Overrun are considered to be \textbf{Disrupted}. Only defending units can become Disrupted and Disruption pertains only to a Overrun - not normal combat. The status of Disruption is indicated by placing a Disrupted Marker on top of the unit(s).

\subsubsection{} Disrupted units may not attack; they do defend normally. They may \textbf{not} move, nor do they exert a Zone of Control. Additional "Disruption" results and/or normal combat results have not further Disruption effect.

\subsubsection{} Disrupted units return automatically to normal in the applicable Friendly Disruption Removal Phase.

\subsubsection{} If a Soviet Leader becomes Disrupted it may not function as a Leader, although it continues to function as a normal - if Disrupted - combat unit. Disrupted Soviet Leaders may not coordinate supply and/or combat for regular Soviet combat units, nor may they aid Soviet attacks.

\subsection{TERRAIN EFFECTS CHART}
