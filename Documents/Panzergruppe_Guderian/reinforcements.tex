\section{REINFORCEMENTS}

GENERAL RULE:

Both sides receive reinforcements according to the Reinforcement Schedules for each side (Section 16.0). In addition, the Soviet Player receives Provisional Reinforcements each Turn, and he may, if he chooses, bring in reinforcements from his South Western Front Armies. Reinforcements pay the necessary Movement Point costs to enter the first hex on the game map, and all reinforcements are considered to be in supply for the First \textbf{Player}-Turn of their entrance. All Soviet reinforcements arrive in an Untried state.

CASES:

\subsection{SOVIET PROVISIONAL\\*REINFORCEMENTS}

\subsubsection{} On each Turn, starting with Game Turn One, the Soviet Player throws a die. The number thrown corresponds to one of the six Soviet Provisional Reinforcement Entrance hexes or areas on the game map. These hexes/areas are numbered one through six. The Soviet Player receives one \textbf{Rifle} Division (Untried) in that Turn, arriving on the hex or in the area corresponding to the number he has rolled on the die. Thus, if the Soviet Player rolls a "three" he receives one Rifle Division at Entrance Hex Three \textbf{in addition} to his normal reinforcements (16.1).

\subsubsection{} If the Entrance Hex or Area is completely occupied by German units, or in German Zones of Control, the Soviet Provisional Reinforcements may enter the game map at the hex nearest to that hex which is unoccupied in the direction of the eastern edge of the board.

\subsubsection{} The Soviet Player \textbf{must} take the Provisional Reinforcements in the Turn in which they are to appear; they may not be saved or accumulated.

\subsubsection{} The counter limitation is absolute; the Soviet Player may never have more units in play than there are counters provided. He may, however, make use of units tht had been previously destroyed (see Case 12.4). These units must still be brought back in an Untried state.

\subsection{SOVIET SOUTH-WESTERN FRONT REINFORCEMENTS}

The Soviet Player may choose to divert units from the (off-map) South-Western Front to aid in his defense of Moscow. In doing so, he will add to the German Victory Point Total; the Soviet Player will have to weigh the pros and cons of such a decision.

\subsubsection{} Beginning with Game-Turn Two, the Soviet Player may bring on as many as five Rifle Divisions per Game-Turn, up to a maximum of ten divisions for the entire game, as South Western Front Reinforcements. These reinforcements are in addition to any regularly scheduled reinforcements as well as the Provisional Reinforcements. The Soviet Player is never required to bring on any of these divisions; but the maximum maybe be no more than five per Turn and ten per game.

\subsubsection{} South-Western Front Reinforcements may enter the game map through any Railroad hex on the \textbf{southern} edge of the map east of, and including, Entrance Hex "Z".

\subsubsection{} For each of the first five divisions that the Soviet Player chooses to bring in, the German Player receives \textbf{one} Victory Point \textbf{each}. For each of the next five divisions the Soviet Player may choose to use, the German Player receives \textbf{two} Victory Points \textbf{each}. If the Soviet Player were to bring in all ten divisions, the German Player would receive 15 Victory Points.

\subsubsection{} There is no Leader provided with these reinforcements.

\begin{flushleft}
  \subsection{SCHEDULED REINFORCEMENTS}
\end{flushleft}

Both sides receive scheduled reinforcements. These are listed in Section 16.0. These reinforcements may be delayed at the whim of the Players. However, they must be brought onto the map in the hex or area specified. For example, units scheduled to enter in area B may enter in hex 0101, 0102, 0103, 0104, 0105, 0106, 0107, 0108, 0109, 0110, or 0112. The Turn shown is the First Game-Turn in which the reinforcements listed may be brought into the game.