\section{INTERDICTION}

COMMENTARY:

Control of the skies by the Luftwaffe enabled the Germans to restrict, to some extent, troop movements by the Soviets along major arteries. The Soviets, in return, had a feeble air force and a minimal rear-guard partisan effort.

CASES:

\subsection{GERMAN AIR\\*INTERDICTION MARKERS}

The German Player receives three Air Interdiction Markers. These represent the effects of tactical bombing raids on various communications points in the Soviet Union. They are only Markers and they have no Zone of Control, combat capabilities or any other characteristics of combat units. They do not affect stacking.

\subsection{USE OF\\*GERMAN AIR INTERDICTION}

The German Player places his Air Interdiction Markers on the game map during his Air Interdiction Phase. However, before the game begins (and after the Soviet Player has placed his initial units on the game map) the German Player places down his Air Interdiction Markers for the First Soviet Player-Turn (see section 5.0). Air Interdiction Markers may be placed no further east than the 4000 hexrow (inclusive) unless the German Player controls both hexes of Smolensk. One turn \textbf{after} he takes Smolensk, and if a Line of Communications (see Case 15.1) of any length connects it to the western map edge, he may place his Air Interdiction Markers anywhere on the game map. If Smolensk is recaptured by the Soviets, the German Player reverts to the original restrictions.

\subsection{EFFECTS OF GERMAN AIR INTERDICTION}

\subsubsection{} If an Air Interdiction Marker is placed on a Railroad hex, it costs each \textbf{four} of the RR Bonus Movement Points to move through that hex, if they are using Rail Movement. Otherwise, it \textbf{adds one} Movement Point to the cost of other terrain in the hex.

\subsubsection{} If an Air Interdiction Marker is placed on any other hex on the game map, it \textbf{adds one} Movement Point to the cost for Russian units entering that hex.

\subsubsection{} No more than one Air Interdiction Marker may be placed in any one hex. All Markers are \textbf{removed} at the end of the Soviet Movement Phase. Air Interdiction Markers have \textbf{no effect} on the tracing of Soviet Supply. Soviet Supply may be traced through terrain with Air Interdiction Markers.

\subsubsection{} German Interdiction Markers may be placed in Soviet-occupied hexes.

\subsection{SOVIET INTERDICTION}

The Soviet Player has sporadic ability to interfere with German movement and supply. This minimal ability rises from small partisan groups and remnants of the Soviet Air Force.

\subsubsection{} On any three Game-Turns in the game, excluding the last Turn (Game-Turn Twelve), and only on three Turns, the Soviet Player may use a Soviet Interdiction Marker.

\subsubsection{} The Interdiction Marker is placed on the game map at the \textbf{end} of the Soviet Player-Turn. It may be placed anywhere on the game map and is not counted against stacking. It has no Zone of Control or any effect other than that listed below.

\subsubsection{} The Soviet Interdiction Marker has the same effect on German units as the German Air Interdiction Marker has on Soviet units as per Cases 13.31 and 13.32. This effect lasts through \textbf{both} German Movement Phases.

\subsubsection{} In addition, the German Player may not trace Supply through the hex occupied by the Soviet Interdiction Marker. The Soviet Interdiction Marker interrupts any German Supply Lines.

\subsubsection{} The Soviet Interdiction Marker may not be placed in a hex occupied by a German unit.

\subsubsection{} The Soviet Interdiction Marker is removed at the end of the German Player-Turn.