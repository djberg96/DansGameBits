\begin{flushleft}
  \section{HOW TO PLAY THE GAME}
\end{flushleft}

\textbf{SETTING UP THE GAME}

The game begins with the Soviet Player placing his initial forces on the game map. All Soviet units, except for Leaders, start the game in an "Untried" state; you cannot tell what their Combat Strength is, although you can see how far they can move (their Movement Allowance). It is only when a Soviet unit becomes involved in combat or an Overrun that it is turned over to reveal its true Combat Strength. German units do not have an Untried state; you can always tell their Strengths. The German Player now places his Air Interdiction Markers on the game map; these Markers tend to slow down Soviet movement. After all units have been placed on the game map (the German Player does not start the game with any combat units on the game map), the Soviet Player begins his Turn.

\textbf{THE SOVIET PLAYER MOVES}

First, the Soviet Player checks to see whether his combat units are in supply. Those units that are out of supply may move only half their Movement Allowance. The Soviet Player funnels supply through his Leaders, who are - among other things - responsible for coordinating Soviet Supply. After determining which units are in, or out of supply, the Soviet Player begins to move.

The Soviet Player may move as many of his units as he wishes in each Turn. They may move as many hexes as they have Movement Points in their Movement Allowance, although some hexes are more expensive to enter or cross than others. For example, it costs a unit an extra Movement Point to cross a River hexside. The Soviet Player may also choose to speed up his movement for some of his units by using Railroads. If any Soviet unit moves into a hex next to a German unit, it must stop; it has entered the German unit's Zone of Control. The Soviet Player continues his Movement portion of his Turn until he has moved all the units he wishes.

During movement, the Soviet Player may want to try to "Overrun" German units. He can do this in his Movement Phase by expending three Movement Points and dividing his Attack Strength in half. If the Overrun attack works and the German Player has to retreat or is eliminated, the Soviet Player may keep on moving until he uses all his Movement Points. This Overrun is considered part of movement, although it may seem like combat. Units that are Overrun become Disrupted, or virtually useless for an entire Turn.

\textbf{THE SOVIET PLAYER ATTACKS}

Actual combat occurs after all movement has ceased. The Soviet Player may now attack any German units that are adjacent to his own units, as long as his units are within range of a Leader. He does not have to attack, and he may choose which units he wishes to attack with. It is here that the Untried units are turned over and their Strengths revealed (if the unit turns out to be a "0-0-6" it is removed from play). The Soviet Player then totals up his Combat Strength Points and compares them to the Strength of the German units for each attack he is making. He then reduces this comparison to a simple odds ratio, such as "2 to 1", "3 to 1", etc., and uses a die and the Combat Results Table to find out what happened.

The Combat Results Table tells the Players whether the units in an attack have to take a loss or retreat. The Players follow the instructions of the CRT after each combat, and the Player who wins each individual battle may get to advance into any vacated hexes. After determining the resutls of all combat, the Soviet Player then removes any Disruption Markers from his units. He also removes the German Air Interdiction Markers, possibly placing down his own. His Turn is over.

\textbf{THE GERMAN PLAYER MOVES}

The German Player now checks the Reinforcement Chart to see which of his units come into the game - and where. He also checks for Supply the same way as the Soviet Player. However, the German Player does not have Leaders; rather, his units trace Supply directly to the western edge of the game map. The German Player then moves his units as did the Soviet Player, conducting any Overruns he might wish to try.

\textbf{THE GERMAN PLAYER ATTACKS}

After his Movement Phase, the German Player attacks any Soviet units to which he is adjacent, as did the Soviet Player. However, the German Player gets a bonus if he has all the units from the same Panzer or Motorized Division in the same hex. If this occurs, the division has its Combat Strength doubled.

\begin{flushleft}
  \textbf{THE GERMAN PLAYER MOVES AGAIN}
\end{flushleft}

Unlike the Soviet Player, the German Player gets another chance to move after completing his combat. In this second Movement Phase, the German Player may move any of his Cavalry, Panzer, Mech or Motorized units as he wishes, checking for Supply again. They may conduct Overruns, etc, just like in the first Movement Phase. After the German Player completes this second Movement Phase, he, too, removes any Disruption markers and places his Air Interdiction Markers on the game map for the next Soviet Game-Turn.

\textbf{IN SUMMARY}

The Players will find that it will aid the flow of the game immensely if they keep an eye on the Sequence of Play (4.0). The Sequence of Play is the focal point of the game, as it informs the Players what functions they must perform and in what order. In essence, it is the skeletal structure upon which the game hangs.

The above sequence is followed, in general for \textbf{twelve} Game-Turns, after which the Players check the Victory Conditions to see who has won.

It is best to set up the game map now, before reading further; punch out all the counters and check the section on the Initial Set-Up. Place the Soviet counters on the game map, and as you read the rules push the counters around to get the feel of the game. Relax, read through the rules and see how they work. Note any questions you have as you go along; they're probably answered later on in another section. In no time you'll be back at the Eastern Front.