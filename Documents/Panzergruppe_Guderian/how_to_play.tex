\section{HOW TO PLAY THE GAME}

\subsection{SETTING UP THE GAME}

The game begins with the Soviet Player placing his initial forces on the game map. All Soviet units, except for Leaders, start the game in an "Untried" state; you cannot tell what their Combat Strength is, although you can see how far they can move (their Movement Allowance). It is only when a Soviet unit becomes involved in combat or an Overrun that it is turned over to reveal its true Combat Strength. German units do not have an Untried state; you can always tell their Strengths. The German Player now places his Air Interdiction Markers on the game map; these Markers tend to slow down Soviet movement. After all units have been placed on the game map (the German Player does not start the game with any combat units on the game map), the Soviet Player begins his Turn.

\subsection{THE SOVIET PLAYER MOVES}

First, the Soviet Player checks to see whether his combat units are in supply. Those units that are out of supply may move only half their Movement Allowance. The Soviet Player funnels supply through his Leaders, who are - among other things - responsible for coordinating Soviet Supply. After determining which units are in, or out of supply, the Soviet Player begins to move.

The Soviet Player may move as many of his units as he wishes in each Turn. They may move as many hexes as they have Movement Points in their Movement Allowance, although some hexes are more expensive to enter or cross than others. For example, it costs a unit an extra Movement Point to cross a River hexside. The Soviet Player may also choose to speed up his movement for some of his units by using Railroads. If any Soviet unit moves into a hex next to a German unit, it must stop; it has entered the German unit's Zone of Control. The Soviet Player continues his Movement portion of his Turn until he has moved all the units he wishes.

During movement, the Soviet Player may want to try to "Overrun" German units. He can do this in his Movement Phase by expending three Movement Points and dividing his Attack Strength in half. If the Overrun attack works and the German Player has to retreat or is eliminated, the Soviet Player may keep on moving until he uses all his Movement Points. This Overrun is considered part of movement, although it may seem like combat. Units that are Overrun become Disrupted, or virtually useless for an entire Turn.