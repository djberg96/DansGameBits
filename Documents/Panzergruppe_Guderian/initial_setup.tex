\section{INITIAL SET-UP}
GENERAL RULE:
The Soviet Player places his initial units on the map first, in accordance with the placement listed below. After the Soviet Player has placed all his initial units, the German Player places his three Air Interdiction Markers in accordance with Case 12.2. The game then begins.

The Soviet Player always moves first in each Game-Turn. Note that there are no German units (except for Interdiction Markers) on the game map during the Soviet Player-Turn. German Air Interdiction Markers are placed on the map before the die roll specified in Case 5.21 is attempted.

Punch out all Soviet combat units, and turn them face-down so that their Untried side is showing. The Soviet Player now randomly chooses the number of units specified, making certain to choose the type called for. All the units are then stacked in the hex listed. Stacking restrictions must not be exceeded at the end of the First Soviet Movement Phase. If more than one hex is listed, the Soviet units may be placed in any of the hexes, within stacking restrictions (see Case 12.1).

\subsection{SOVIET INITIAL DEPLOYMENT}
\textbf{In hex 4015}:
The 24th Army (5 Rifle Divisions, 1 Armored Division, General \textbf{Rakutin}).

\textbf{In hex 2216}:
The 16th Army (6 Rifle Divisions, General \textbf{Lukin}).

\textbf{In hex 1414}:
The 19th Army (4 Rifle Divisions, General \textbf{Konlev}).

\textbf{In hex 0123, 0124, 0125 and/or 0126:}
The 13th Army (7 Rifle Divisions, 1 Armored Division, General \textbf{Remezov}).

\textbf{In hexes 0108, 0109, 0110, 0111, 0112, 0113, 0114, and/or 0115:}
The 20th Army (6 Rifle Divisions, 4 Armored Divisions, General \textbf{Kurochkin}).

\begin{flushleft}
  \subsection{SOVIET FIRST TURN SPECIAL RULES}
\end{flushleft}

The Soviet Player is somewhat restricted as to what he may do on his First Player-Turn.

\subsubsection{} The die must be rolled once for each Army to determine whether the 16th Army (2216) and the 19th Army (1414) may be moved on Turn One. If the Soviet Player rolls a "1", "2" or "3", the army in question may move. Any number other than those given results in that army standing in place for the entire Turn (see Case 5.22). Both armies are free to move on Game-Turn Two.

\subsubsection{} If the 16th and/or the 19th Army is unable to move, it must still satisfy stacking restrictions by the end of the Movement Phase. To do so, the Player may, if necessary, move as many units as are needed one hex (maximum) to satisfy the stacking restrictions.

\subsubsection{} All units of the 13th and 20th Armies \textbf{must expend their full Movement Allowance in the First Game-Turn}. They may not end the Movement Phase of the First Game-Turn in either of the two westernmost hexrows (0100 and 0200) on the game map, nor may they enter any hex more than once. Moreover, they may never move in a westerly direction; they may move north, south, or east only. Neither the 13th nor the 20th Army may use Rail Movement on the First Game-Turn.